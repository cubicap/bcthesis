\shorthandoff{'}
\begin{markdown*}{%
  hybrid,
  definitionLists,
  footnotes,
  inlineFootnotes,
  hashEnumerators,
  fencedCode,
  citations,
  citationNbsps,
  pipeTables,
  tableCaptions,
}

\chapter{Jaculus-machine} \label{chap:machine}

Jaculus-machine is a standalone, embeddable, C++ centric JavaScript runtime. It uses the QuickJS JavaScript engine at its core and provides a C++ interface, with the aim of providing a simple and easy-to-use API for adding features to the runtime.

A large portion of Jaculus-machine is a set of wrapper classes around the QuickJS API. The classes provide RAII semantics and an easy-to-use API for routine use cases.

Jaculus-machine uses two core concepts, which are referred to as *Machine* and *MFeature* in the code and the rest of this thesis. A *Machine* is an instance of the runtime complete with all the selected MFeatures. An *MFeature* is a module that can be used as a part of a Machine and which provides functionality to the runtime or to other MFeatures.

The modular design of Jaculus-machine allows for easy extensibility and customization of the runtime. The user can select which MFeatures should be included in the runtime and also create new MFeatures with a low effort. It also allows for easier portability of existing MFeatures to other platforms.

All public functionality of the library is contained in the `jac` namespace. QuickJS exports its symbols in the global namespace, and their names are usually prefixed with `JS`. The user should not interact with QuickJS directly for regular use, as the library provides wrappers for most of its functionality.

\section{Machine and MFeatures}

The main entry point of the library is a Machine instance. After a Machine is created and initialized, it can be used to interact with the JavaScript runtime and execute JavaScript code.

A Machine is defined using templated stack inheritance from the `MachineBase` class and selected MFeature classes. The `MachineBase` class provides the core functionality of the runtime, and the MFeature classes implement additional functionality, such as an event loop and filesystem access.

Stack inheritance is a technique where a class is defined as a template parametrized by another class, which it then inherits from. The inheritance is resolved at compile time, and each class in the stack can access the members of the classes lower in the stack. An example of stack inheritance can be seen in a code snippet below.

```cpp
class A { ... };

template<class Next>
class B : public Next { ... };

template<class Next>
class C : public Next { ... };

using Stack = C<B<A>>;
```

The stack design of the Machine allows interfacing with different MFeatures in C++ directly without any middleware. Lower-level MFeatures are located lower in the stack and implement platform-specific functionality, while higher-level MFeatures are located higher in the stack and can use the abstraction provided by the lower-level MFeatures. This allows for easy portability of higher-level MFeatures to other platforms.

In other words, instead of the MFeatures being initialized in relative isolation, as in many other runtimes, they are initialized as a part of a Machine, strictly in the context of lower-level MFeatures.

\section{JavaScript concepts in C++}

Because many JavaScript concepts do not have a direct equivalent in C++ or have different semantics, Jaculus-machine provides a set of classes that wrap their representation from the QuickJS API. The classes provide RAII semantics and aim to be as easy to use as possible.

\subsection{Values}

JavaScript values can be either primitive values (e.g., numbers) or objects (e.g., arrays). Primitive values are stored in stack memory, while objects are stored in the heap. QuickJS uses reference counting for memory management, and the user is responsible for decreasing the reference count of JavaScript values that are no longer needed. Jaculus-machine provides a set of classes that wrap JavaScript values and provide RAII semantics for them. The classes also provide a simple API for converting to and from C++ types.

The base class for JavaScript value wrapper is `ValueWrapper` and provides a general API for working with JavaScript values. More specific wrapper classes are derived from `ValueWrapper` and provide additional functionality, such as `ObjectWrapper` and `FunctionWrapper`.

`ValueWrapper` has a template parameter `managed` that specifies whether the wrapper takes ownership of the underlying JavaScript value. This pattern is assumed from QuickJS, which sometimes does not give ownership of the value to the user to reduce the number of changes in its reference count. If the value is a reference, depending on the value of `managed`, the wrapper will either be a strong or a weak reference to the value. For convenience, the library provides two type aliases for all built-in wrappers, which are usually used instead and follow the following pattern:

```cpp
// value/strong reference
using Value = ValueWrapper<true>;
// value/weak reference
using ValueWeak = ValueWrapper<false>;
```

A `ValueWrapper` instance can be obtained either as a result of the execution of JavaScript code or by creating it from C++. New JavaScript values can be created using static methods of the `ValueWrapper` and its subclasses. The following code shows some examples:

```cpp
ContextRef ctx = ...;

Value undefined = Value::undefined(ctx);
Value number = Value::from<int>(ctx, 42);
Value string = Value::from<std::string>(ctx, "Hello, world!");
Value object = Object::create(ctx);
Value array = Array::create(ctx);
```

The `ValueWrapper` class provides methods for converting the value to a C++ value or a subclass of `ValueWrapper` to access additional functionality. The following code shows some examples:

```cpp
Value value = ...;

int number = value.to<int>();
std::string string = value.to<std::string>();
Object object = value.to<Object>();
```

If the wrapped value cannot be converted to the requested type, a `jac::Exception` will be thrown.

\subsection{Type conversions}

In many cases, the library can perform automatic type conversions between JavaScript values and C++ types. Among others, automatic conversions happen in getters, setters, and function calls. This allows for wrapping existing functions without having to write conversion code manually.

Default conversions for several built-in types are provided, such as `int`, `double`, `std::string`, and `std::vector`. The library also provides a mechanism for defining custom conversions for user-defined (and not-yet-supported built-in) types. To define a custom conversion, the user must implement a template specialization of the `ConvTraits` structure for the type. The following code shows a specialization for the `bool` type, which is included with the library:

```cpp
template<>
struct ConvTraits<bool> {
  static bool from(ContextRef ctx, ValueWeak val) {
    return JS_ToBool(ctx, val.getVal());
  }

  static Value to(ContextRef ctx, bool val) {
    return Value(ctx, JS_NewBool(ctx, val));
  }
};
```

\subsection{Exceptions}

A portion of the Jaculus-machine library consists of wrappers that allow calling C++ functions from JavaScript code. When an exception is thrown in the wrapped C++ code, it is caught by the wrapper, converted to a JavaScript exception, and propagated to the runtime. The library allows the user to specify the JavaScript `Error` type or any other JavaScript value that should be thrown.

Similarly, when a JavaScript function is called from C++, and an exception is thrown in the runtime, it is caught by the wrapper, converted to a C++ exception of type `jac::Exception`, and propagated to the C++ code.

Different types of C++ exceptions are converted to JavaScript exceptions as follows:

  - `jac::Exception` --- a specified `Error` type, or the wrapped value if `type == Type::Any`
  - `std::exception` --- an `InternalError` object with the message `e.what()`
  - any other exception --- an `InternalError` object with the message "unknown exception"

To create an `Exception` that will be converted to JavaScript as a specified `Error` type, the `Exception::create` method can be used:

```cpp
Exception::create(Exception::Type::TypeError, "invalid argument");
```

To create an `Exception` that will be converted to JavaScript as a specified value, the value should be created independently and then converted to an `Exception` using the `ValueWrapper::to` method:

```cpp
Value value = ...;
Exception exception = value.to<Exception>();
```

\subsection{Functions}

The library provides an interface for defining JavaScript functions by wrapping most existing callable C++ objects. This interface is presented in the form of `FunctionFactory` class.

To wrap a function, the user can use the `newFunction` and `newFunctionThis`. All arguments that are passed to the function call will be converted to the types of the function parameters. If the number of arguments does not match or if the values cannot be converted to the required types, a `TypeError` will be thrown.

Because variadic functions in C++ are processed at build time, they can not be universally wrapped to create their JavaScript counterparts. For this reason, `FunctionFactory` lets the user define a function with an argument of type `std::vector<ValueWrapper>`. When this function is called from JavaScript, the vector will contain all arguments passed to the function. These function can be created using the `newFunctionVariadic` and `newFunctionThisVariadic` methods.

The methods `newFunctionThis` and `newFunctionThisVariadic` additionally give access to the `this` value of the function (for example, when the function is called as a method of an object).

The following code shows some examples:

```cpp
ContextRef ctx = ...;
FunctionFactory ff(ctx);

Function add = ff.newFunction([](int a, int b) { return a + b; });

Function sum = ff.newFunctionVariadic([](std::vector<ValueWeak> args) {
  int sum = 0;
  for (auto& arg : args) {
    sum += arg.to<int>();
  }
  return sum;
});
```

JavaScript functions can be called from C++ using the methods `Function::call` and `Function::callThis`. The latter additionally gives access to the `this` value of the function. Both methods have a template argument that specifies the type the return value should be converted to; arguments are converted to JavaScript values automatically. Object constructors in JavaScript are also represented as functions but have different call semantics --- in JavaScript, they are called with the `new` keyword. To call a constructor from C++, the `Function::callConstructor` method should be used. The following code shows some examples using the functions defined above:

```cpp
int added = add.call<int>(1, 2);  // returns 3
int summed = sum.call<int>(1, 2, 3, 4);  // returns 10

Function ctor = machine.eval("class { constructor(a) { this.a = a; } }");
Value res = ctor.callConstructor(42);  // returns { a: 42 }
```


\section{Features}

Jaculus-machine mainly consists of wrappers around QuickJS API and provides only a minimum set of MFeatures that are required for the runtime to function correctly or are helpful for development and debugging. Instead, the library aims to provide functionality, which makes the creation of new MFeatures as easy as possible.

\subsection{Built-in MFeatures}

The following MFeatures are included with the library (their class names are suffixed with "Feature" in the library):

  - EventQueue --- an event queue that can be used to schedule events to be executed in the event loop
  - EventLoop --- an event loop that executes events from an event queue in the main thread
  - Filesystem --- an abstraction over the filesystem that provides access to files and directories
  - ModuleLoader --- an implementation of module loader for importing modules from the filesystem (using the `import` statement in JavaScript) and evaluating JavaScript files (using the `evalFile` method in C++)
  - BasicStream --- an implementation of simple readable and writable stream types
  - Stdio --- a feature adding `stdin`, `stdout`, and `stderr` streams to the Machine and `console` interface to the JavaScript side
  - Timers --- typical JavaScript timers and a sleep function

\subsection{Default event loop}

Default event loop implementation is split into two separate MFeatures. One implements an event queue, and the other implements the event loop. This allows for easier portability, as the event loop itself can be reused on different platforms, while the event queue can be extended to, for example, support scheduling events from interrupts.

\subsection{Watchdog}

The `MachineBase` class implements a watchdog timer that can be used to detect infinite loops in the runtime. The watchdog can be configured using the `setWatchdogTimeout` method. By setting the timeout to 0, the watchdog can be disabled, which is the default state.

By default, the watchdog will interrupt the runtime on timeout. This behavior can be changed by setting the watchdog handler using the `setWatchdogHandler` method. Instead of interrupting, the handler will be called, and depending on its return value, the runtime will either be interrupted or continue running.

\subsection{Class wrapping} \label{sub:class-wrapping}

Sometimes, it is desirable to create a JavaScript object backed by a C++ object. This can be done using the class `Class`. The class is templated by a `ProtoBuilder` structure, which defines how the JavaScript object prototype should be constructed and how the optional opaque data should be managed. The class `Class` can then be used to initialize the class in a given Context, construct the JavaScript object or obtain its constructor.

An example use case could be a class that represents a device peripheral. The C++ object could contain a handle to the peripheral, implement functionality for using the peripheral, and use the JavaScript object as a proxy to access these functions from JavaScript code.


\section{Usage} \label{sec:machine-usage}

The central part of the library is the `Machine` type, which represents a JavaScript runtime with a set of MFeatures. An example definition of a Machine is shown below:

```cpp
using Machine = ComposeMachine<
  ModuleLoaderFeature,
  FilesystemFeature,
  StdioFeature,
  BasicStreamFeature,
  MachineBase
>;
```

The `ComposeMachine` class is a helper class that builds the Machine inheritance chain. The MFeatures at the beginning of the argument list will be at the top of the stack, and the MFeatures at the end will be at the bottom. The `MachineBase` class is a base class for all Machines and must be at the end of the list.

To use the machine, an instance of the `Machine` type must be created. Then the machine can be configured and after calling the `initialize` method, it can be used. The following code shows an example of creating and initializing a Machine:

```cpp
Machine machine;

// Configure the machine
machine.setWatchdogTimeout(1000);
machine.setWatchdogHandler([](Machine& machine) {
  std::cerr << "Watchdog timeout!" << std::endl;
  return true;
});
machine.initialize();

// Use the machine
machine.evalFile("main.js");
```


\subsection{JavaScript classes}

As described in Section \ref{sub:class-wrapping}, the library provides a mechanism for creating JavaScript classes, which can be backed by C++ objects.

To create a class, the user must first define a `ProtoBuilder` structure. The `ProtoBuilder` describes the class's behavior and properties through its static interface. Its features are specified by inheriting from the structures in the `ProtoBuilder` namespace and overriding their static interfaces. These structs contain a default implementation of their interface and some convenience functions for describing the class. The following code shows an example of a `ProtoBuilder`:

```cpp
struct MyBuilder : public ProtoBuilder::Opaque<MyClass>, public ProtoBuilder::Properties {
  static MyClass* constructOpaque(ContextRef ctx, std::vector<ValueWeak> args) {
    return new MyClass();
  }

  // The default destructOpaque uses delete. By overriding it, a custom deleter can be used

  static void addProperties(ContextRef ctx, Object proto) {
    addPropMember<int, &MyClass::foo>(ctx, proto, "foo");
  }
};
```

The `Class` template can be instantiated with the `ProtoBuilder` structure to create a class definition, and a name can be assigned using the `init` method. The `init` method can be called repeatedly without any effect if called with the same name; otherwise, an exception will be thrown. To use the class in a given Context, the class must be initialized in the Context by calling the `initContext` method:

```cpp
// Initialize the class
using MyClassJs = Class<MyBuilder>;
MyClassJs::init("MyClass");

// Initialize the class in a context
ContextRef ctx = ...;
MyClassJs::initContext(ctx);
```

After doing the above steps, the class constructor and prototype can be retrieved and the class can be instantiated:

```cpp
Value constructor = MyClassJs::getConstructor(ctx);
Value prototype = MyClassJs::getPrototype(ctx);

Value obj = constructor.to<Function>().callConstructor();
```

\subsection{Creating new Features}

An MFeature is defined as a template class. To allow building the inheritance chain of a Machine, the class must set the class `Next` presented in its last template parameter as its base class. To add functionality to the machine, the class may:

  - present a public C++ interface to MFeatures higher in the inheritance chain or to the user of the Machine, and
  - use the `initialize` method to add functionality to the JavaScript runtime

In the MFeature constructor, the user must not interact with the JavaScript runtime in any way, as it is not initialized yet. Initialization of the MFeature involving the JavaScript runtime should be done in the `initialize` method.

The following code shows an example of a Feature that adds a `foo` property to the global object:

```cpp
template<typename Next>
class MyFeature : public Next {
  void initialize() {
    ContextRef ctx = this->context();
    Object global = ctx.getGlobalObject();
    global.defineProperty("foo", Value::from(ctx, 42));
  }
};
```

The user might also want to define a JavaScript module. This can be done using the `MachineBase::newModule` method. The following code creates a module named `myModule`, which exports a value named `foo`:

```cpp
template<typename Next>
class MyFeature : public Next {
  void initialize() {
    ContextRef ctx = this->context();

    Module& module = this->newModule("myModule");
    module.addExport("foo", Value::from(ctx, 42));
  }
};
```


\shorthandon{'}
\end{markdown*}
