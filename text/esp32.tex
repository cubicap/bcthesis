\shorthandoff{'}
\begin{markdown*}{%
  hybrid,
  definitionLists,
  footnotes,
  inlineFootnotes,
  hashEnumerators,
  fencedCode,
  citations,
  citationNbsps,
  pipeTables,
  tableCaptions,
}

\chapter{Jaculus-esp32}

By adding hardware bindings to Jaculus-dcore, it is possible to create a firmware for a specific platform. The version created as a part of this thesis is Jaculus-esp32.

Jaculus-esp32 is a Jaculus device firmware for the ESP32 and ESP32-S3 microcontrollers. It uses the ESP-IDF framework and supports connection to a computer via a serial port.

\section{Features}

Aside from the features provided by Jaculus-dcore, Jaculus-esp32 only implements control over the most basic peripherals, which are:

  - GPIO -- general-purpose input/output pins
  - ADC -- analog-to-digital converter
  - LEDC -- generator of PWM signals
  - Neopixel -- WS2812B smart LED strip

Jaculus-esp32 also implements a specialized event queue based on a FreeRTOS queue and supports scheduling events from an interrupt context. This is used in the GPIO feature, which generates events when the state of a pin changes through an interrupt.


\section{Usage}

The firmware can be flashed to the device manually using ESP-IDF tools or the Jaculus CLI tool described in the next chapter.

The firmware uses the Jaculus-machine ((TODO: link)) runtime at its core, which means that adding new features to the runtime is done the same way as described in ((TODO: link)).

\shorthandon{'}
\end{markdown*}
