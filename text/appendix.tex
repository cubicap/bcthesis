\shorthandoff{'}
\begin{markdown*}{%
  hybrid,
  definitionLists,
  footnotes,
  inlineFootnotes,
  hashEnumerators,
  fencedCode,
  citations,
  citationNbsps,
  pipeTables,
  tableCaptions,
}

\appendix

\chapter{Attachments}

All source code created as a result of this thesis is available in the attached ZIP archive. The archive contains the following directories, each containing the source code of the respective part of the implementation:

- `Jaculus-dcore`
- `Jaculus-machine`
- `Jaculus-link`
- `Jaculus-esp32`
- `Jaculus-tools`
- `QuickJS`

The `Jaculus-esp32/ts-examples` directory contains examples of TypeScript programs that can be run on the Jaculus-esp32 firmware.

Documentation for `Jaculus-machine` and `Jaculus-link` is available in the `docs` subdirectory of their respective directories. It is also hosted online at \url{https://machine.jaculus.org}, and \url{https://link.jaculus.org}.


\chapter{Building Jaculus for ESP32}

The Jaculus-esp32 firmware can be built using the ESP-IDF version 5.0.1 or newer. For the setup of the ESP-IDF, please refer to the official documentation\footnote{In: ESP-IDF Programming Guide \[online\]. \[visited on 2023-05-16\]. Available from: \url{https://docs.espressif.com/projects/esp-idf/en/v5.0.1/esp32/get-started/}}.

The ESP-IDF can be used from the command line. To set up the environment variables for the ESP-IDF, run the corresponding export script in the ESP-IDF directory:

- Linux, macOS: `source export.sh`
- Windows: `export.bat` or `export.ps1`

Select the target platform by renaming the corresponding configuration file in the `Jaculus-esp32` directory of the project to `sdkconfig`:

- ESP32: `sdkconfig-esp32`
- ESP32-S3: `sdkconfig-esp32s3`

To build the firmware, run `idf.py build` in the `Jaculus-esp32` directory. To flash the firmware to the device, run `idf.py flash`.

Note that the build process needs to fetch the Jaculus-dcore, Jaculus-machine, Jaculus-link, and QuickJS dependencies from GitHub and requires an internet connection. To install the dependencies manually, copy their source directories to the `Jaculus-esp32/components` directory:

- \texttt{Jaculus-dcore/src} \rightarrow \texttt{components/jaculus-dcore}
- \texttt{Jaculus-machine/src} \rightarrow \texttt{components/jaculus-machine}
- \texttt{Jaculus-link/src} \rightarrow \texttt{components/jaculus-link}
- \texttt{quickjs} \rightarrow \texttt{components/quickjs}


\chapter{Building Jaculus-tools}

Jaculus-tools is developed in TypeScript and requires Node.js version 18 or newer. To build the application, first install its development and runtime dependencies by running `npm ci` in the `Jaculus-tools` directory. Then, build the application by running `npm run build`.

The application can be run using `npx jac` in the `Jaculus-tools` directory. To install the application globally, run `npm link` in the `Jaculus-tools` directory. This will create a symbolic link to the application in the global `node_modules` directory, allowing it to be run from anywhere using `npx jac` (or `jac` if the global `node_modules` directory is in the `PATH` environment variable).


\shorthandon{'}
\end{markdown*}
