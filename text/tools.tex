\shorthandoff{'}
\begin{markdown*}{%
  hybrid,
  definitionLists,
  footnotes,
  inlineFootnotes,
  hashEnumerators,
  fencedCode,
  citations,
  citationNbsps,
  pipeTables,
  tableCaptions,
}

\chapter{Jaculus-tools}

To allow the user to interact with Jaculus devices, a command-line application called Jaculus-tools was created. The application is implemented in TypeScript and is available as an npm package under the name `jaculus-tools`.

The package could also be used as a library for building custom applications, but it is not documented. The user might want to look at the source code of the command-line application part of the package for inspiration on how to use the library.

\section{Features}

The application provides commands to check the status of the device (`status`, `version`), install the Jaculus firmware (`install`), control the JavaScript runtime (`start`, `stop`, `monitor`), and access the device's filesystem (`ls`, `read`, `write`, `rm`, `mkdir`, `rmdir`, `upload`). The application also provides commands for uploading and downloading code to the device (`build`, `flash`, `pull`).

Other utility commands are also available:

  - `help` -- prints help for the specified command
  - `list-ports` -- lists available serial ports
  - `serial-socket` -- tunnel a serial port over a TCP socket

The commands are run by specifying them after the `jac` command.

```
$ jac <command> [options] [arguments]
```

\section{Implementation}

\subsection{Connection to the device}

The application implements a simplified version of the interface described in ((TODO - Jaculus-link)) -- the `Router` class is omitted, as connecting to multiple devices is mostly pointless. Aside from that and language choice, the implementation is almost identical.

\subsection{Device access}

The application provides access to the device via the `Device` class. The class exposes the following functionality to the user:

  - device lock
  - Controller and Uploader services
  - standard input and output of the running program
  - output of the device logger

\subsection{Command-line argument parser}

The application uses a custom parser for command-line arguments, which allows for chaining compatible commands.

The commands can also access and modify a global state object passed to each command, allowing them to share data --- for example, once the device is connected, the `Device` instance is saved to the state object and can be accessed by other commands.

For example, the following command:

```
$ jac --port /dev/ttyUSB0 build flash monitor
```

Will sequentially run the three specified commands (`build`, `flash`, `monitor`). The `build` command compiles the code and saves the compiled code to `build` directory. The `flash` command connects to the device, saves the device to the parser state, and uploads the code. The `monitor` command uses the saved device to access its standard input and output without reconnecting.

Command-line options are divided into two types -- global and command-specific. Options can be specified in any place of the command, as the parsing is done in multiple passes for each specified command. The global options are parsed first, then options of the first command are extracted, then of the second, and so on. Each command then receives only its and the global options. This forces the commands to use different names for their options, as otherwise, preceding commands might extract them. For example, the following two commands would be parsed the same way:

```
$ jac --port /dev/ttyUSB0 build flash monitor
$ jac build flash monitor --port /dev/ttyUSB0
```

The commands can also specify standard arguments, which are parsed after the options:

```
$ jac --port /dev/ttyUSB0 read ./code/index.js
```


\subsection{TypeScript code compilation}


\shorthandon{'}
\end{markdown*}
