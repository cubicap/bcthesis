\shorthandoff{'}
\begin{markdown*}{%
  hybrid,
  definitionLists,
  footnotes,
  inlineFootnotes,
  hashEnumerators,
  fencedCode,
  citations,
  citationNbsps,
  pipeTables,
  tableCaptions,
}

\chapter{Jaculus-tools}

To allow the user to interact with Jaculus devices, a command-line application called Jaculus-tools was created. The application is implemented in TypeScript and is available as a package in the npm registry\footnote{npm registry contains packages for the Node.js runtime environment. The registry can be accessed through a website at \url{https://www.npmjs.com/} or through the `npm` command-line tool.} under the name `jaculus-tools`.

The package can also be used as a library for building custom applications, but this functionality is currently not documented. The potential user might want to look at the source code of the command-line application part of the package for inspiration on how to use the library.

\section{Features} \label{sec:tools-features}

The application provides commands to check the status of the device (`status`, `version`), install Jaculus firmware to the device (`install`), control the JavaScript runtime (`start`, `stop`, `monitor`), and access the device's filesystem (`ls`, `read`, `write`, `rm`, `mkdir`, `rmdir`, `upload`, `pull`). The application also provides commands for compiling and uploading code to the device (`build`, `flash`).

Other utility commands are also available:

  - `help` -- prints help for the specified command
  - `list-ports` -- lists available serial ports
  - `serial-socket` -- tunnel a serial port over a TCP socket

The commands are run by specifying them after the `jac` command.


\section{Implementation}

\subsection{Communication with the device}

To communicate with the device, the application must implement the same protocol as the Jaculus-link library described in Section \ref{sec:mux-protocol}.

The application implements a similar pipeline to the one used in the library, except for the Routing layer, which is omitted, as connecting to multiple devices would be pointless. Aside from that and the language choice resulting in some different programming paradigms, the architecture is mostly similar.

\subsection{Device access}

The application provides access to the device via the `Device` class. The class exposes the following functionality to the user:

  - Controller and Uploader services
  - standard input and output of the running program
  - output of the device logger

\subsection{Command-line argument parser}

The application uses a custom parser for command-line arguments, allowing for chaining compatible commands.

The commands can also access and modify a global state object passed to each command, allowing them to share data --- for example, once the device is connected, the `Device` instance is saved to the state object and can be accessed by other commands.

For example, the following command:

```
jac --port /dev/ttyUSB0 build flash monitor
```

Will sequentially run the three specified commands (`build`, `flash`, `monitor`). The `build` command compiles the code and saves the compiled code to `build` directory. The `flash` command connects to the device, saves the device to the global state, and uploads the code. The `monitor` command uses the saved device to access its standard input and output without reconnecting.

Command-line options are divided into two types --- global and command-specific. Options can be specified in any place of the command, as the parsing is done in multiple passes for each specified command. The global options are parsed first, then options of the first command are extracted, then the second, and so on. Each command then receives only its and the global options. For example, the following two commands would be parsed the same way:

```
jac --port /dev/ttyUSB0 build flash monitor
jac build flash monitor --port /dev/ttyUSB0
```

The commands can also specify positional arguments, which are parsed after the options:

```
jac --port /dev/ttyUSB0 read ./code/index.js
```

\subsection{TypeScript code compilation}

To allow the user to write their code in TypeScript, the command-line tool uses the TypeScript compiler to compile the code. Microsoft provides the compiler as an npm package called `typescript`, which can be used programmatically. The compiler is used to compile the code to JavaScript, which is then uploaded to the device.

The structure of a TypeScript project is described in the `tsconfig.json` file, which is loaded by the compiler. The file specifies, among other things, which files should be compiled, where the compiled files should be saved, and which ECMAScript version should be used for the output code.

Because the target runtime is known to the compiler, some options in the `tsconfig.json` file are constrained in preprocessing, such as the target ECMAScript version. If the user specifies an invalid value of these options, the compiler will throw an error.


\section{Usage}

The application is available as an npm package, which can be installed using the `npm` command-line tool:

```
npm install -g jaculus-tools
```

The `-g` option installs the package globally, which makes the `jac` command available in the terminal\footnote{The path to the global `npm` directory must be added to the `PATH` environment variable for the `jac` command to be available. Otherwise, it can be run using `npx jac`.}.

The commands listed in Section \ref{sec:tools-features} can be run using the `jac` command.

```
jac <command> [options] [arguments]
```

The `help` command can be used to get help for a specific command:

```
jac help <command>
```


\shorthandon{'}
\end{markdown*}
