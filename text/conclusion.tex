\shorthandoff{'}
\begin{markdown*}{%
  hybrid,
  definitionLists,
  footnotes,
  inlineFootnotes,
  hashEnumerators,
  fencedCode,
  citations,
  citationNbsps,
  pipeTables,
  tableCaptions,
}


\chapter{Evaluation}

Several requirements were outlined in the introduction. This section evaluates the solution based on those requirements.

Compared to native programs, the solution significantly reduces the development cycle length. While the development cycle of native programs can, in worse cases, take several minutes, with Jaculus, it is reduced to several seconds. The build time shows the most considerable improvement, as no build process is needed for JavaScript programs, and the build time of TypeScript programs is mostly negligible compared to native programs. Deployment is also faster, as only the application code needs to be uploaded to the device, unlike native programs, where the entire firmware partition is often overwritten.

The shortened development cycle naturally comes at a price, that being the performance of JavaScript as an interpreted language. However, the performance is still sufficient for running most application logic. If a performance-sensitive part of the application is identified, it can be reimplemented in C++ and compiled into a native module.

\section{Comparison with other solutions}

There are already existing solutions to programming microcontrollers using JavaScript. While it would be nice to have a detailed comparison of at least some of them, it would stretch the scope of this thesis too far. However, a brief comparison of Jaculus with other solutions is provided in this section. These solutions were chosen based on their popularity:

- CircuitPython\cite{circuitpython} (fork of MicroPython\cite{micropython})
- NodeMCU\cite{nodemcu} (Lua)
- Espruino\cite{espruino} (JavaScript)
- Moddable SDK\cite{moddable} (JavaScript)

A common characteristic of these solutions is their much larger feature set compared to Jaculus. They have not only much larger hardware and library support but also larger ecosystems and communities. Espruino and CircuitPython provide very convenient development environments.

What sets Jaculus apart from the other solutions is its much higher-level abstraction around different JavaScript concepts. This allows for much easier wrapping of existing libraries and other code to JavaScript modules, meaning that while the hardware support of Jaculus is currently low, adding more features to the runtime is a quick and straightforward task.

\subsection{Non-JavaScript solutions}

Comparing different languages, especially their performance, is a notoriously difficult task, as each language has its own strengths and weaknesses. Therefore, the comparison focuses on the usability of the solutions.

MicroPython and CircuitPython are implementations of Python~3. They are very popular in the maker community and have a large ecosystem of libraries and tools. However, they only support a limited subset of the language, meaning that some Python code cannot be run on them, which could be seen as a disadvantage.

NodeMCU is a Lua interpreter for ESP8266 and ESP32 microcontrollers. On the one hand, it is a well-established project with an existing community. On the other hand, as mentioned in the introduction, Lua is not nearly as popular as the other languages, making it less attractive to new users.

\subsection{JavaScript solutions}

Comparing Jaculus with the other JavaScript solutions is more straightforward, as they all share the same language. Their main difference is their performance, implementation, and tooling.

Espruino is an open-source JavaScript interpreter for microcontrollers. The biggest problem of Espruino is its low performance, which can be seen in the performance comparison below. Espruino also implements only a limited subset of an older version of the ECMAScript standard, meaning some JavaScript code cannot be run on it. Espruino also provides a convenient development environment through its Web IDE, which allows for live code editing and debugging and does not require any additional tools to be installed.

Moddable SDK is another JavaScript runtime for microcontrollers. It is a commercial product but is open-source and free for non-commercial use. Instead of interpreting the JavaScript code on the device, Moddable SDK compiles it into bytecode on the host machine, packs necessary native libraries, and builds a complete firmware image, which is then uploaded to the device. This approach allows for a smaller runtime footprint but causes a slightly longer development cycle and requires much more complex development tools.

The performance of these solutions was compared by running the same benchmarks on the ESP32 microcontroller. Aside from the GPIO benchmark, the benchmarks were taken from the Computer Language Benchmarks Game\cite{bench-game} and slightly modified for each platform to fit the provided API. The GPIO benchmark was created for this comparison and measured the time it takes to toggle a GPIO pin *n* times. The code of the GPIO benchmark for Espruino is shown below. The results of the comparison are shown in Table \ref{tab:performance}.

```js
var pin = 5;
var start = Date.now();
for (var i = 0; i < n; i++) {
  digitalWrite(pin, 1);
  digitalWrite(pin, 0);
}
var end = Date.now();
console.log("Time: " + (end - start));
```

The benchmarks show that Jaculus is significantly faster in computational tasks than Espruino and Moddable SDK. Moddable SDK is faster in the binary-trees benchmark, which is targeted at the creation of new objects and memory allocation. Moddable SDK is also faster in the GPIO benchmark, which may be caused by poorly optimized abstraction used to implement the native module in Jaculus.

The low performance of Espruino can likely be attributed to the fact that internally, Espruino parses the JavaScript code into an abstract syntax tree, which is then directly interpreted. By contrast, the approach used by Jaculus and Moddable SDK, is first to compile the source code into bytecode, which is then interpreted. This approach allows for more optimizations to be performed on the bytecode, which results in better performance.


\begin{table}[ht]
  \centering

  | Test                  | Jaculus     | Espruino    | Moddable SDK |
  |-----------------------|------------:|------------:|-------------:|
  | n-body (n=50)         | 36.2        | 4333.1      | 54.4         |
  | fannkuch-redux (n=6)  | 35.6        | 12335.3     | 75.0         |
  | spectral-norm (n=10)  | 48.9        | 10749.9     | 91.4         |
  | binary-trees (n=3)    | 279.3       | 19930.4     | 187.6        |
  | GPIO (n=5000)         | 756.1       | 11720.4     | 54.7         |

  \caption[Performance comparison of Jaculus with other JavaScript solutions]{Performance comparison of Jaculus with other JavaScript solutions. The results are run times in milliseconds, lower is better.}

  \label{tab:performance}
\end{table}


\chapter{Limitations}

This chapter describes the limitations of the created components and the Jaculus-esp32 firmware. While the problems described here are not critical, they cause some inconvenience and should be addressed in the future.

\section{Only one Context per Machine instance}

From the beginning, the design of Jaculus-machine was to have only one Context per Machine instance. It seemed not to be a limiting factor and simplified implementation. However, later in development, it started to show it was a wrong decision and proves to be a limiting factor in some cases.

For example, in REPL, all exceptions should be caught and reported to standard output. When starting REPL from a JavaScript program, the main program should crash on unhandled exceptions, whereas the REPL should not. Implementation of this would require REPL and the main program to be executed in separate contexts to distinguish between their behavior regarding exception handling.


\section{Unhandled promise rejections not being reported}

This is a limitation of QuickJS. Although QuickJS does have a mechanism for reporting unhandled promise rejections, it reports some false positives. Consider the following example:

``` js
new Promise((resolve, reject) => {
  console.log("promise");
  reject(null);
}).then(() => {
  console.log("ok");
}).catch(() => {
  console.log("error");
});

console.log("after");
```

The promise is created and immediately rejected. At that moment, the promise does not have a rejection handler, and thus QuickJS reports an unhandled promise rejection. However, the handler is added before the promise goes out of scope and handles the rejection.

Because of the false positives, the mechanism for reporting unhandled promise rejections is disabled in Jaculus-machine.

Fortunately, this is not a problem for well-written code, which correctly handles all possible promise rejections. However, when an unhandled promise rejection occurs, it is not reported, which may lead to errors that are difficult to debug.

To fix this, modifying QuickJS internals would be required, which is outside the scope of this theis, but it is an essential consideration for future work.


\section{Filesystem API}

As the Filesystem MFeature is implemented using C++ `std::filesystem` API, which is not yet fully supported by the ESP-IDF, some of its functionality does not work, and some may even block the runtime indefinitely.

The broken functionality involves listing directories --- `readdir` and `rmdir`. Fortunately, working with files works without a problem, which is an essential functionality for loading JavaScript files.

It would be possible to separately reimplement the MFeature to use the POSIX API, but it would make more sense to implement a more complex abstraction layer around the filesystem API, which could be used by other Jaculus components as well. However, that would require a significant amount of work, which is not strictly necessary for the functionality of Jaculus-esp32.

\section{Compatibility with other platforms}

While implementing all of the created components, I have focused on only using the standard C++20 library. This worked well for development, as all of the functionality could be tested locally on a desktop PC, and later everything could be easily integrated with the ESP-IDF. Unfortunately, after briefly exploring development options of platforms other than ESP32 (STM32, RP2040), I have discovered that the C++20 standard is often not fully supported. Some platforms do not even fully support older standards, such as C++17.

This dealt a significant blow to the portability of these components, which would have to be partially rewritten to support other platforms. Mainly, an abstraction layer around the filesystem API and asynchronous elements (threads, synchronization primitives) would have to be implemented.


\chapter{Conclusion}

The goal of this thesis was to create an ecosystem for programming embedded devices using JavaScript.

The presented solution consists of Jaculus-dcore library and Jaculus-tools command-line application. The library provides the core functionality of a Jaculus device, and the application provides a way to interact with the device. The Jaculus-dcore library is also integrated into the Jaculus-esp32 firmware, which ports the solution to the ESP32 platform.

Two standalone libraries were also created as part of the solution: Jaculus-link and Jaculus-machine. The former is a communication library, and the latter is an implementation of the JavaScript runtime with easy extensibility. Both libraries are well documented and tested, and can be used independently from the rest of the solution.

Although the solution is in a usable state and, in terms of performance, can compete with other existing solutions, many features are still missing, and some bugs are inevitably present. Most importantly, more MFeatures must be implemented for Jaculus to be a viable alternative to other existing solutions. Further improvements can also be made to the upload protocol, which is presently very simple and could have better performance and reliability.

\shorthandon{'}
\end{markdown*}
