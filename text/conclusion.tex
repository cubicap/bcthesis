\shorthandoff{'}
\begin{markdown*}{%
  hybrid,
  definitionLists,
  footnotes,
  inlineFootnotes,
  hashEnumerators,
  fencedCode,
  citations,
  citationNbsps,
  pipeTables,
  tableCaptions,
}


\chapter{Evaluation}

Several requirements were outlined in the introduction. This section evaluates the solution based on those requirements.

Compared to native programs, the solution significantly reduces the length of the development cycle. While the development cycle of native programs can, in worse cases, take several minutes, with Jaculus, it is reduced to several seconds. The build time shows the largest improvement, as no build process is needed for JavaScript programs, and the build time of TypeScript programs is mostly negligible in comparison with native programs. Deployment is also faster, as only the application code is uploaded to the device, in contrast to native programs, where the entire firmware partition is usually overwritten.

The shortened development cycle naturally comes at a price; that being the performance of JavaScript as an interpreted language. However, the performance is still sufficient for running most application logic, and if a performance-sensitive part of the application is identified, it can be reimplemented in C++ and compiled into a native module.

\section{Comparison with other solutions}

There are already existing solutions to programming microcontrollers using JavaScript and while it would be nice to have a detailed comparison of at least some of them, it would expand the scope of this thesis too much. However, a brief comparison of Jaculus with other solutions is provided in this section. These solutions were chosen based on their popularity:

- CircuitPython (fork of MicroPython)
- NodeMCU (Lua)
- Espruino (JavaScript)
- Moddable SDK (JavaScript)

A common point of all of these solutions is their much larger feature set compared to Jaculus. They have not only much larger hardware and library support but also larger ecosystems and communities. Espruino and CircuitPython provide very convenient development environments.

What sets Jaculus aside from the other solutions is its much higher-level abstraction around different JavaScript concepts. This allows for much easier wrapping of existing libraries and other code to JavaScript modules, meaning that while the hardware support of Jaculus is currently low, adding more features to the runtime is a quick and simple task.

\subsection{Non-JavaScript solutions}

Comparing different languages, and especially their performance, is a notoriously difficult task, as each language has its own strengths and weaknesses. The comparison is therefore focused on the usability of the used languages.

MicroPython and CircuitPython are implementations of Python 3. They are very popular in the maker community and have a large ecosystem of libraries and tools. However, they only support a limited subset of the language, which means that some Python code cannot be run on them, which could be seen as a disadvantage.

NodeMCU is a Lua interpreter for ESP8266 and ESP32 microcontrollers. On one hand, it is a well-established solution with an existing community. On the other hand, as mentioned in the introduction, Lua by itself is not nearly as popular as other languages, which makes it less attractive for developers.

\subsection{JavaScript solutions}

The comparison of Jaculus with the other JavaScript solutions is more straightforward, as they all share the same language. The main difference between them is their performance and the way they are implemented.

Their performance was compared by running the same computational benchmarks on the ESP32 microcontroller. The results are shown in table ((TODO)).


Espruino is a JavaScript interpreter for microcontrollers and supports a wide range of devices, including the ESP32. The biggest problem of Espruino is its low performance. It is caused by the fact, that internally, Espruino parses the JavaScript code into an abstract syntax tree, which is then directly interpreted, instead of compiling it into bytecode.

Moddable SDK is a JavaScript runtime for microcontrollers. It is a commercial product, but it is open-source and free for non-commercial use. ((TODO: approach))


\chapter{Conclusion}

The goal of this thesis was to create an ecosystem for programming embedded devices using JavaScript.

The solution consists of Jaculus-dcore library and Jaculus-tool command-line application. The library provides the core functionality of a Jaculus device, and the application provides a way to interact with the device. The Jaculus-dcore library is also integrated into the Jaculus-esp32 firmware, which ports the solution to the ESP32 platform.

Two standalone libraries were also created as part of the solution: Jaculus-link and Jaculus-machine. The former is a communication library, and the latter is an implementation of the JavaScript runtime with easy extensibility. Both libraries are well documented and tested, and can be used independently of the rest of the solution.

Although the solution is in usable state, there are still some features missing and some bugs present. Most importantly, more MFeatures need to be implemented for Jaculus to be a viable alternative to other existing solutions. Further improvements can also be made to the upload protocol, which is currently very simple and could have better performance.

\shorthandon{'}
\end{markdown*}
