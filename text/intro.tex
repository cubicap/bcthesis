\shorthandoff{'}
\begin{markdown*}{%
  hybrid,
  definitionLists,
  footnotes,
  inlineFootnotes,
  hashEnumerators,
  fencedCode,
  citations,
  citationNbsps,
  pipeTables,
  tableCaptions,
}

\chapter{Motivation}

Microcontrollers in embedded devices are typically programmed in compiled languages, such as C, C++, or Rust. Although these languages provide high performance and low overhead at runtime, they can be difficult to learn and use.

Another problem of compiled languages in embedded environments is a long development cycle. Because of the limited resources of a microcontroller, the executable of such applications is usually self-contained, and, besides the application itself, contains all used libraries, including the standard ones. All of the libraries must be linked on every build and in some cases also compiled regardless of whether they have changed. The development cycle is often further prolonged by long deployment times. The build-deploy process can take several minutes, severely hindering development speed, especially in the early development stages when the developer rapidly iterates over the code.

Teaching embedded programming to beginners is a big motivator for reducing the development cycle length. Seeing the results of their work quickly is important for keeping them motivated and interested in the subject, especially with children and their increasingly short attention span.

As embedded devices often interact with the physical world or communicate with other devices, they often have to wait for external events rather than actively perform some computation. Therefore, most embedded applications have reactive and asynchronous elements, which are difficult to express in low-level languages. This induces a lot of boilerplate code, which obfuscates the main application logic and makes it harder to develop and maintain. Even though high-level languages such as C++ and Rust somewhat alleviate this issue, C is often the only language with direct support from the manufacturers, adding more work for the developer.


\chapter{Proposed solution}

The proposed solution is to use a high-level, interpreted language instead. The language has to:

  - have low enough hardware requirements to run on a microcontroller
  - be easy to learn and use
  - be able to express reactive and asynchronous elements
  - be embeddable into C/C++ applications

The solution should also provide an ecosystem for managing the device and developing applications. The ecosystem should include:

  - a firmware for the microcontroller, which gives complete control over the device to the runtime environment and provides an API for managing the device
  - a command-line application for managing the device and uploading code to it

\subsection{Language choice}

Many interpreted languages are available. Because of their popularity in embedding into other applications, the following languages were considered:

  - Python
  - Lua
  - JavaScript

Python is a general-purpose language with an extensive standard library, which makes it suitable for many applications. However, it is not suitable for embedded platforms due to its high memory requirements caused by its large standard library.

Lua is a lightweight scripting language with a small standard library. It is suitable for embedded devices and there are Lua interpreters, which are embeddable into C/C++ applications. On the other hand, Lua is not nearly as popular as Python and JavaScript, making it harder to find relevant resources and justify learning a new language.

JavaScript is a popular language commonly used in web browsers. The language specification defines a small standard library, which is extended by the runtime environment, meaning a small memory footprint. It also maps very well to the event-centered nature of many embedded systems, as JavaScript is inherently event-driven. Multiple embeddable JavaScript engines exist, such as DukTape, MuJS, and QuickJS.

For the above reasons, JavaScript was chosen to be used in the final solution.

Because JavaScript is weakly typed, debugging errors caused by type mismatches is sometimes challenging. A possible solution is to use a strongly typed language, such as TypeScript. However, as compiling or running TypeScript on a microcontroller is not feasible, it must first be transpiled\footnote{Transpilation is the process of source-to-source compilation} to JavaScript.


\chapter{Overview}

The main task is to create a JavaScript runtime environment for microcontrollers and an ecosystem around it for managing the device and developing applications for it.

The runtime should primarily focus on the ESP32 and ESP32-S3 microcontrollers, as they are prevalent in the maker community and provide high performance at a low price.

A firmware for a microcontroller, which gives complete control over the device to the JavaScript runtime and provides an interface for programming and controlling the device, should be created. A device running this firmware will be called a *Jaculus device*.

\section{Runtime environment}

The runtime environment should be able to run JavaScript code and be easily extensible with functionality implemented in C++. The runtime should be usable as the main interface for programming the device and as a component of a larger application.

An example use case for the latter is in a system of devices used as game elements. Their low-level logic (e.g., communication, user interface) can be implemented in C++, while the high-level logic (e.g., game rules) can be updated independently by the user.

\section{JavaScript engine}

A JavaScript engine is needed to interpret JavaScript code. Implementing a custom JavaScript engine would be a significant undertaking, so an existing one is used instead. There are multiple options, the popular ones being V8, DukTape, MuJS, QuickJS.

The V8 engine is the most popular JavaScript engine, used in Google Chrome and Node.js. It is a high-performance engine with a very large memory footprint, making it impossible to run on a microcontroller.

DukTape and MuJS are small, embeddable JavaScript engines. They are suitable for embedded devices but support old versions of the ECMAScript specification.

QuickJS is a small, embeddable JavaScript engine that supports the ECMAScript 2020 specification. On a desktop platform, it is 2-4 times faster ((TODO)) than DukTape and MuJS, for the price of a larger memory footprint, which is still small enough to run on a microcontroller.

The final solution uses QuickJS. Compared to the other options, it stands on a nice middle ground regarding its feature set, performance, and memory footprint.

\section{Communication}

There should be a way to communicate with the device --- to upload code, control the runtime and monitor the device's state.

Most microcontrollers feature a serial interface, such as a USB to UART bridge or a native USB interface. A serial interface, however, only provides a single duplex byte stream, meaning a protocol must be implemented on top of it to support multiple services over a single connection.

Using a single stream connection also adds flexibility in the choice of the transport medium. Aside from the serial interface, the protocol can be used over a network socket, web socket, or any other byte stream connection.

\section{Tooling}

A suitable tooling should be created to support the development of applications for the device. The tools should allow the user to upload code to the device, control the runtime, and monitor the device's state.

\section{Implementation}

To achieve the goals described above and to allow possible future reuse of independent components, the project is split into multiple parts:

  - Jaculus-machine -- standalone, embeddable, C++ centric JavaScript runtime using QuickJS at its core
  - Jaculus-link -- standalone communication library for multiplexing multiple channels on a single stream connection
  - Jaculus-dcore -- core library for creating new Jaculus devices
  - Jaculus-tools -- command-line application for controlling and monitoring Jaculus devices
  - Jaculus-esp32 -- Jaculus device port for the ESP32 platform (with support for ESP32 and ESP32-S3 SOCs)


\chapter{Used technologies}

This chapter briefly describes some selected technologies used in the project and some of their specifics.

\section{ESP32 and ESP32-S3}

ESP32 and ESP32-S3 are microcontrollers by Espressif Systems.

ESP32 is a dual-core microcontroller based on the Xtensa LX6 CPU architecture with 520 KiB of SRAM. It features a UART interface for programming.

ESP32-S3 is a dual-core microcontroller based on the Xtensa LX7 CPU architecture with 512 KiB of SRAM. It features a native USB interface and a UART interface for programming.

Both of these microcontrollers also provide a Wi-Fi and Bluetooth interface and a large number of other peripherals. On many boards based on these microcontrollers, the programming UART interface is connected to a USB-to-UART bridge, which allows the microcontroller to be programmed over USB. The USB ports are also often used for communication with the program running on the microcontroller.

\section{ESP-IDF}

ESP-IDF is the official development framework for microcontrollers from Espressif Systems. The framework is based on FreeRTOS and provides a set of libraries and tools for developing applications for, besides others, ESP32 and ESP32-S3 microcontrollers.

Most of the provided libraries have only a C API. The framework also supports C++20 with a large subset of its standard library. However, some parts of the standard library do not work entirely correctly (e.g., `std::filesystem`).

\section{JavaScript}

JavaScript is a high-level, interpreted, weakly, and dynamically typed programming language. It is standardized in the ECMAScript specification, which is maintained by Ecma International.

Although JavaScript programs are event-driven, the code is executed in a single thread. This is achieved by using an event loop, where asynchronous events are queued and executed in the order they are received. Therefore, JavaScript programs must be written in a non-blocking manner, as blocking the event loop will cause the program to stop responding to events. Events are generated by the JavaScript engine or the host environment both synchronously and asynchronously.

\section{TypeScript}

TypeScript is a strongly typed superset of JavaScript. It is developed and maintained by Microsoft. TypeScript is typically not interpreted directly and is instead compiled into JavaScript, which can be interpreted using any JavaScript runtime that supports the specified ECMAScript version.

\section{QuickJS}

QuickJS\cite{quickjs} is a JavaScript engine implementing the ECMAScript 2020 specification\cite{es2020} (ES2020). It was developed by Fabrice Bellard and is licensed under the MIT license. It is written in C and is designed to be embeddable in other applications.

QuickJS uses POSIX to implement atomic operations and system time. Although this slightly limits its portability, ESP-IDF, the primary target platform for Jaculus, supports POSIX.

According to ES2020, JavaScript code is evaluated in a *Realm*, which defines the execution environment (e.g., global object and set of built-in objects). QuickJS uses a different term for this concept --- Context, which I have adopted for Jaculus and will be used throughout the rest of this thesis.


\shorthandon{'}
\end{markdown*}
