\documentclass[
  digital,
  oneside,
  nosansbold,
  nocolorbold,
  nolof,
  nolot
]{fithesis4}

\usepackage[resetfonts]{cmap}
\usepackage[T1]{fontenc}
\usepackage[
  main=english,
  english, czech, slovak
]{babel}

\usepackage{paratype}

\thesissetup{
    date        = \the\year/\the\month/\the\day,
    university  = mu,
    faculty     = fi,
    type        = bc,
    department  = Department of Computer Science,
    author      = Petr Kubica,
    gender      = m,
    advisor     = {RNDr. Jan Mrázek.},
    title       = {Jaculus: Approachable Programming of Embedded Devices via Javascript},
    TeXtitle    = {Jaculus: Approachable Programming of Embedded Devices via Javascript},
    keywords    = {embedded systems, embedded programming, ESP32, JavaScript, TypeScript},
    abstract    = {
        This thesis presents the design and implementation of Jaculus, a platform for programming microcontrollers using JavaScript and TypeScript. The platform consists of a runtime for the microcontroller and a command-line tool for accessing the device. The runtime is built around the QuickJS JavaScript engine and provides a high-level abstraction over JavaScript concepts to make creating new native features as easy as possible. The command-line tool can be used to connect to the device, upload source code, and control the device.

        The runtime is then ported to the ESP32 and ESP32-S3 series of microcontrollers with several native modules. The port is then used to demonstrate the usage of Jaculus and is evaluated against other similar solutions.
    },
    thanks      = {
        I would like to thank everyone who helped me make this thesis possible, especially my advisor Jan Mrázek, for his guidance and advice. I would also like to thank everyone from Robotárna for their support, feedback, and constant encouragement. Not to be forgotten are my friends, who for months had to constantly listen to me talking about Jaculus\footnote{\scriptsize{I am sorry, it is going to happen again.}} and, nevertheless, helped me proofread this thesis. Finally, I would like to thank my family for their support and bottomless patience.
    },
    bib         = bibliography.bib,
    facultyLogo = fithesis-fi,
}
\usepackage{makeidx}
\makeindex
\usepackage{paralist}
\usepackage{url}
\usepackage{markdown}

\usepackage[htt]{hyphenat}
\def\markdownRendererCodeSpan#1{\texttt{#1}}
\usepackage{listings}
\lstset{
  basicstyle       = \footnotesize\ttfamily,
  identifierstyle  = \color{black},
  keywordstyle     = \color{blue},
  keywordstyle     = {[2]\color{cyan}},
  keywordstyle     = {[3]\color{olive}},
  stringstyle      = \color{red},
  commentstyle     = \itshape\color{teal},
  breaklines       = true,
  showstringspaces = false
}
\usepackage{floatrow}
\floatsetup[table]{capposition=top}
\usepackage[babel]{csquotes}


\lstdefinelanguage{js}{
  keywords={typeof, new, true, false, catch, function, return, null, catch, switch, var, if, in, while, do, else, case, break},
  ndkeywords={class, export, boolean, throw, implements, import, this},
  comment=[l]{//},
  morecomment=[s]{/*}{*/},
  morestring=[b]',
  morestring=[b]",
}

\lstdefinelanguage{cpp}{
  keywords={alignas, alignof, and, and_eq, asm, atomic_cancel, atomic_commit, atomic_noexcept, auto, bitand, bitor, bool, break, case, catch, char, char8_t, char16_t, char32_t, class, compl, concept, const, consteval, constexpr, constinit, const_cast, continue, co_await, co_return, co_yield, decltype, default, delete, do, double, dynamic_cast, else, enum, explicit, export, extern, false, float, for, friend, goto, if, inline, int, long, mutable, namespace, new, noexcept, not, not_eq, nullptr, operator, or, or_eq, private, protected, public, reflexpr, register, reinterpret_cast, requires, return, short, signed, sizeof, static, static_assert, static_cast, struct, switch, synchronized, template, this, thread_local, throw, true, try, typedef, typeid, typename, union, unsigned, using, virtual, void, volatile, wchar_t, while, xor, xor_eq},
  comment=[l]{//},
  morecomment=[s]{/*}{*/},
  morecomment=[l][\color{violet}]{\#},
  morestring=[b]',
  morestring=[b]",
}


\begin{document}

\shorthandoff{'}
\begin{markdown*}{%
  hybrid,
  definitionLists,
  footnotes,
  inlineFootnotes,
  hashEnumerators,
  fencedCode,
  citations,
  citationNbsps,
  pipeTables,
  tableCaptions,
}

\chapter{Motivation}

Microcontrollers in embedded devices are typically programmed in compiled languages, such as C, C++, or Rust. Although these languages provide high performance and low overhead at runtime, they can be difficult to learn and use.

Another problem of compiled languages in embedded environments is a long development cycle. One cause is that the executable of such applications is usually self-contained. It contains all used libraries, including the standard ones, besides the application itself, which must be compiled and linked on every build, regardless of whether they have changed. Another cause is that uploading the executable to the device memory is often slow. The build-deploy process can take several minutes, severely hindering development speed, especially in the early development stages when the developer rapidly iterates over the code.

As embedded devices often interact with the physical world or communicate with other devices, they often have to wait for external events rather than actively performing some computation. Therefore, most embedded applications have reactive and asynchronous elements, which are difficult to express in low-level languages. This induces a lot of boilerplate code, which obfuscates the main application logic and makes it harder to develop and maintain. Even though high-level languages such as C++ and Rust somewhat alleviate this problem, C is often the only language with direct support from the manufacturers, adding more work for the developer.


\chapter{Proposed solution}

The proposed solution is to use a high-level, interpreted language instead. The language has to:

  - have low enough hardware requirements to run on a microcontroller
  - be easy to learn and use
  - be able to express reactive and asynchronous elements
  - be embeddable into C/C++ applications

Three popular options are Python, Lua, and JavaScript.

Python is a general-purpose language with an extensive standard library, which makes it suitable for many applications. However, it is not suitable for embedded devices due to its high memory requirements caused by the large standard library.

Lua is a lightweight scripting language with a small standard library. It is suitable for embedded devices but not as popular as Python and JavaScript, making it harder to find relevant resources and justify learning a new language. There are interpreters of Lua, which are embeddable into C/C++ applications.

JavaScript is a popular language commonly used in web browsers. The language specification defines a small standard library, which is extended by the runtime environment, meaning a small memory footprint. It also maps very well to the event-centered nature of many embedded systems, as JavaScript is inherently event-driven. Multiple embeddable JavaScript engines exist, such as DukTape, MuJS, and QuickJS.

For the above reasons, JavaScript was chosen to be used in the final solution.

Because JavaScript is weakly typed, debugging errors caused by type mismatches is sometimes challenging. A possible solution is to use a strongly typed language, such as TypeScript. However, as compiling or running TypeScript on a microcontroller is not feasible, it must first be transpiled to JavaScript.


\chapter{Overview}

The main task is to create a JavaScript runtime environment for microcontrollers and an ecosystem around it for managing the device and developing applications for it.

The runtime should primarily focus on the ESP32 and ESP32-S3 microcontrollers, as they are prevalent in the maker community and provide high performance at a low price.

A firmware for a microcontroller, which gives complete control over the device to the JavaScript runtime and provides an interface for programming and controlling the device, should be created. A device running this firmware will be called a *JavaScript device*.

\section{Runtime environment}

The runtime environment should be able to run JavaScript code and be easily extensible with functionality implemented in C++. The runtime should be usable as the main interface for programming the device and as a component of a larger application.

An example use case for the latter is in a system of devices used as game elements. Their low-level logic (e.g., communication, user interface) can be implemented in C++, while the high-level logic (e.g., game rules) can be updated independently by the user.

\section{JavaScript engine}

A JavaScript engine is needed to interpret JavaScript code. Implementing a custom JavaScript engine would be a significant undertaking, so an existing one is used instead. There are multiple options, the popular ones being V8, DukTape, MuJS, QuickJS.

The V8 engine is the most popular JavaScript engine, used in Google Chrome and Node.js. It is a high-performance engine with a very large memory footprint, making it impossible to run on a microcontroller.

DukTape and MuJS are small, embeddable JavaScript engines. They are suitable for embedded devices but support old versions of the ECMAScript specification.

QuickJS is a small, embeddable JavaScript engine that supports the ECMAScript 2020 specification. On a desktop platform, it is 2-4 times faster ((TODO)) than DukTape and MuJS, for the price of a larger memory footprint, which is still small enough to run on a microcontroller.

The final solution uses QuickJS. Compared to the other options, it stands on a nice middle ground regarding its feature set, performance, and memory footprint.

\section{Communication}

There should be a way to communicate with the device --- to upload code, control the runtime and monitor the device's state.

Most microcontrollers feature a serial interface, such as a USB to UART bridge or a native USB interface. A serial interface, however, only provides a single duplex byte stream, meaning a protocol must be implemented on top of it to support multiple services over a single connection.

Using a single stream connection also adds flexibility in the choice of the transport medium. Aside from the serial interface, the protocol can be used over a network socket, web socket, or any other byte stream connection.

\section{Tooling}

A suitable tooling should be created to support the development of applications for the device. The tools should allow the user to upload code to the device, control the runtime, and monitor the device's state.

\section{Implementation}

To achieve the goals described above and to allow possible future reuse of independent components, the project is split into multiple parts:

  - Jaculus-machine -- standalone, embeddable, C++ centric JavaScript runtime using QuickJS at its core
  - Jaculus-link -- standalone communication library for multiplexing multiple channels on a single stream connection
  - Jaculus-dcore -- core library for creating new Jaculus devices
  - Jaculus-tools -- command-line application for controlling and monitoring Jaculus devices
  - Jaculus-esp32 -- Jaculus device port for the ESP32 platform (with support for ESP32 and ESP32-S3 SOCs)


\chapter{Used technologies}

This chapter briefly describes some selected technologies used in the project and some of their specifics.

\section{ESP-IDF}

ESP-IDF is the official development framework for microcontrollers from Espressif Systems. The framework is based on FreeRTOS and provides a set of libraries and tools for developing applications for ESP32 and ESP32-S3 microcontrollers.

Most of the provided libraries have only a C API. The framework also supports C++20 with a large subset of its standard library. However, some parts of the standard library do not work entirely correctly (e.g., `std::filesystem`).

\section{JavaScript}

JavaScript is a high-level, interpreted, weakly, and dynamically typed programming language. It is standardized in the ECMAScript specification, which is maintained by Ecma International.

Although JavaScript programs are event-driven, the code is executed in a single thread. This is achieved by using an event loop, where asynchronous events are queued and executed in the order they are received. Therefore, JavaScript programs must be written in a non-blocking manner, as blocking the event loop will cause the program to stop responding to events. Events are generated by the JavaScript engine or the host environment both synchronously and asynchronously.

\section{TypeScript}

TypeScript is a strongly typed superset of JavaScript. It is developed and maintained by Microsoft. TypeScript is typically not interpreted directly and is instead compiled into JavaScript, which can be interpreted using any JavaScript runtime that supports the specified ECMAScript version.

\section{QuickJS}

QuickJS is a JavaScript engine implementing the ECMAScript 2020 specification. It was developed by Fabrice Bellard and is licensed under the MIT license. It is written in C and is designed to be embeddable in other applications.

QuickJS uses POSIX to implement atomic operations and system time. Although this slightly limits its portability, ESP-IDF, the primary target platform for Jaculus, supports POSIX.

JavaScript code is evaluated in a *Realm* ((TODO: link spec)), which defines the execution environment (e.g., global object and set of built-in objects). QuickJS uses a different term for this concept -- Context, which I have adopted for Jaculus and will be used throughout the rest of this thesis.


\shorthandon{'}
\end{markdown*}


\shorthandoff{'}
\begin{markdown*}{%
  hybrid,
  definitionLists,
  footnotes,
  inlineFootnotes,
  hashEnumerators,
  fencedCode,
  citations,
  citationNbsps,
  pipeTables,
  tableCaptions,
}

\chapter{Jaculus-machine}

Jaculus-machine is a standalone, embeddable, C++ centric JavaScript runtime based on QuickJS. The main goal of Jaculus-machine is to provide a simple and easy-to-use API for adding features to the runtime.

A large portion of Jaculus-machine is a set of C++ classes that wrap the QuickJS API. The classes provide RAII semantics and an easy-to-use API for routine use cases.

Jaculus-machine uses two core concepts, which will be referred to as *Machine* and *MFeature* throughout the rest of this thesis.

  - Machine -- an instance of the runtime complete with all the selected MFeatures
  - MFeature -- a component that can be used as a part of a Machine and which provides functionality to the runtime or to other MFeatures

\section{Architecture}

The main entry point of the library is a Machine object. After a Machine is created and initialized, it can be used to interact with the JavaScript runtime.

\subsection{Machine and MFeatures}

Machine is defined using templated stack inheritance from `MachineBase` and selected MFeature classes. The `MachineBase` class provides the core functionality of the runtime, and the MFeature classes implement additional functionality, such as an event loop, filesystem access, etc.

The stack design of Machine allows interfacing with different MFeatures in C++ directly without any middleware. Lower-level MFeatures are located lower in the stack and implement platform-specific functionality, while higher-level MFeatures are located higher in the stack and can use the abstraction provided by the lower-level MFeatures. This allows for easy portability of higher-level MFeatures to other platforms.

\subsection{Exceptions}

Jaculus-machine provides a set of wrappers that allow calling C++ functions from JavaScript code. When an exception is thrown in the C++ code, it is caught by the wrapper, converted to a JavaScript exception, and propagated to the runtime. The library allows the user to specify the type of JavaScript `Error` or any other JavaScript value that should be thrown.

Similarly, when a JavaScript function is called from C++, and an exception is thrown in the runtime, it is caught by its wrapper, converted to a C++ exception of type `jac::Exception`, and propagated to the C++ code.


\section{Features}

Jaculus-machine mainly consists of wrappers around QuickJS API. The wrappers provide RAII semantics and an easy-to-use API for routine use cases.

The library provides a minimum set of MFeatures that are required for the runtime to function correctly, or are useful for development and debugging. Conversely, the library aims to provide functionality, which makes the creation of new MFeatures as easy as possible.


\subsection{Value wrapping and conversion}

QuickJS uses reference counting for memory management, and the user is responsible for decreasing the reference count of JavaScript values that are no longer needed. Jaculus-machine provides a set of classes that wrap JavaScript values and provide RAII semantics for them. The classes also provide a simple API for converting to and from C++ types.

The base class for JavaScript value wrapper is `ValueWrapper` and provides a general API for working with JavaScript values. More specific wrapper classes are derived from `ValueWrapper` and provide additional functionality, such as `ObjectWrapper` and `FunctionWrapper`.

The wrapped value can be either a value (e.g., a number) or a reference (e.g., an object).
`ValueWrapper` has a template parameter `managed` that defines whether the wrapper takes ownership of the underlying JavaScript value. This pattern is assumed from QuickJS, which sometimes does not give ownership of the value to the user to reduce the number of changes in its reference count. If the value is a reference, depending on the value of `managed`, the wrapper will either be a strong or a weak reference to the value. For convenience, the library provides two type aliases for all built-in wrappers, which are usually used instead and follow the following pattern:

```cpp
// value/strong reference
using Value = ValueWrapper<true>;
// value/weak reference
using ValueWeak = ValueWrapper<false>;
```

Default conversions for several built-in types are provided, such as `int`, `double`, `std::string`, and `std::vector`. The library also provides a mechanism for defining custom conversions for user-defined (and not-yet-supported built-in) types.

Many of these conversions are done automatically in, among others, getters, setters, and function calls. This allows for wrapping existing functions without having to write conversion code manually.

\subsection{Function wrapping}

The library provides an interface for defining JavaScript functions by wrapping almost any existing callable C++ object. This interface is presented in the form of `FunctionFactory` class. Because variadic functions in C++ are processed at build time, they can not be universally wrapped. For this reason, `FunctionFactory` lets the user define a function with an argument of type `std::vector<ValueWrapper>`, which, when called, will contain the arguments passed to the function.

The created wrapper then automatically performs argument and return value type conversions and wraps any exceptions thrown by the wrapped function in a JavaScript exception.

\subsection{Class wrapping}

The library provides an interface for defining JavaScript classes, which can contain opaque C++ objects. The user must first define a `ProtoBuilder` class, which tells the library how to construct the JavaScript object prototype and how to manage the optional opaque data. The class `Class` can then be used to initialize the class in a given Context, construct the JavaScript object or obtain its constructor.

A class created using `Class` can then be used to wrap any existing C++ object and expose it to the JavaScript runtime.

\subsection{Built-in MFeatures}

The following MFeatures are included with the library (their class names are suffixed with "Feature"):

  - EventQueue -- an asynchronous event queue that can be used to schedule events to be executed in the event loop; the events can be scheduled from any thread
  - EventLoop -- an event loop that executes events from an event queue in the main thread
  - Filesystem -- an abstraction over the filesystem that provides access to files and directories
  - ModuleLoader -- an implementation of module loader for loading modules from the filesystem (using the `import` statement in JavaScript) and evaluating JavaScript files (using the `evalFile` function in C++)
  - BasicStream -- an implementation of simple readable and writable stream types
  - Stdio -- a feature adding `stdin`, `stdout`, and `stderr` streams to the Machine and `console` interface only to the runtime
  - Timers -- typical JavaScript timers and a sleep function

\subsection{Watchdog}

The `MachineBase` class provides a watchdog that can be used to detect infinite loops in the runtime. The watchdog can be configured using the `setWatchdogTimeout` method. By setting the timeout to 0, the watchdog can be disabled. The watchdog is disabled by default.

By default, the watchdog will interrupt the runtime on timeout. This behavior can be changed by setting the watchdog handler using the `setWatchdogHandler` method. Instead of interrupting, the handler will be called, and if it returns `true`, the watchdog will interrupt the runtime. If the handler returns `false`, the machine will continue to run, but the watchdog will not be reset.


\section{Implementation}

The library is implemented using C++20 and requires POSIX for QuickJS.

All public functionality is contained in the `jac` namespace. QuickJS exports its symbols in the global namespace, and, in most cases, their names are prefixed with `JS`. In most cases, the user should not interact with QuickJS directly, as the library provides wrappers for most of its functionality.

\subsection{Default event loop}

Default event loop implementation is split into two separate MFeatures. One implements an event queue, and the other implements the event loop. This allows for easier portability, as the event loop itself can be reused on different platforms, while the event queue can be extended to, for example, support scheduling events from interrupts.


\section{Usage}

\subsection{Adding the library to a project}

The library is configured as a CMake project. It exports a CMake target called `jac-machine` that can be linked to other projects.

\subsection{JavaScript values}

As described in ((TODO)), the library provides a mechanism for wrapping JavaScript values.

New JavaScript values can be created using static methods of the `ValueWrapper` and its subclasses. The following code shows some examples:

```cpp
ContextRef ctx = ...;

Value undefined = Value::undefined(ctx);
Value number = Value::from<int>(ctx, 42);
Value string = Value::from<std::string>(ctx, "Hello, world!");
Value object = Object::create(ctx);
Value array = Array::create(ctx);
```

The `ValueWrapper` class provides methods for converting the wrapped value to a C++ value:

```cpp
Value value = ...;

int number = value.to<int>();
std::string string = value.to<std::string>();
```

If the wrapped value cannot be converted to the requested type, a `jac::Exception` will be thrown.

\subsection{JavaScript exceptions}

If wrapped C++ code throws an exception, it will be wrapped to a JavaScript exception and thrown to the JavaScript code. The following exception types are converted as follows:

  - `jac::Exception` -- the exception is converted to a specified `Error` type, or a specified value is thrown
  - `std::exception` -- the exception is converted to an `Error` object with the exception message
  - any other exception -- the exception is converted to an `Error` object with the message "unknown exception"

To create an `Exception` that will be thrown to JavaScript as a specified `Error` type, the `Exception::create` method can be used:

```cpp
ContextRef ctx = ...;

Exception::create(ctx, Exception::Type::TypeError, "invalid argument");
```

To create an `Exception` that will be thrown to JavaScript as a specified value, the value should be created independently and then converted to an `Exception` using the `ValueWrapper::to` method:

```cpp
ContextRef ctx = ...;

Value value = ...;
Exception exception = value.to<Exception>();
```

\subsection{Wrapping C++ functions}

The library provides a mechanism for wrapping C++ functions to be called from JavaScript as described in ((TODO)). C++ functions can be wrapped using the `FunctionFactory` class.

The methods `newFunctionVariadic` and `newFunctionThisVariadic` can be used to create variadic functions -- all arguments that are
passed to the function will be passed in a single `std::vector<ValueWeak>`.

The methods `newFunctionThis` and `newFunctionThisVariadic` additionally give access to the `this` value of the function (for example,
when the function is called as a method of an object)

The following code shows some examples:

```cpp
ContextRef ctx = ...;
FunctionFactory ff(ctx);

Function add = ff.newFunction([](int a, int b) { return a + b; });

Function sum = ff.newFunctionVariadic([](std::vector<ValueWeak> args) {
    int sum = 0;
    for (auto& arg : args) {
        sum += arg.to<int>();
    }
    return sum;
});
```

\subsection{JavaScript classes}

The library provides a mechanism for creating JavaScript classes. These classes can contain opaque C++ data, as described in ((TODO)).

  1. To create a class, the user must first define a `ProtoBuilder` struct. The `ProtoBuilder` describes the class's behavior and properties through its **static** interface. Its features are specified by inheriting from structs from the `ProtoBuilder` namespace and overriding their **static** interfaces. These structs contain a default implementation of their interface and some convenience functions for describing the class. The following code shows some examples:

```cpp
using namespace jac;

struct MyBuilder : public ProtoBuilder::Properties {
    static void addProperties(ContextRef ctx, Object proto) {
        proto.defineProperty("foo", Value::from(ctx, 42));
    }
};

struct MyBuilder2 : public ProtoBuilder::Callable {
    static Value callFunction(ContextRef ctx, ValueWeak funcObj, ValueWeak thisVal, std::vector<ValueWeak> args) {
        return Value::from(ctx, static_cast<int>(args.size()));
    }

    // the default implementation - does not have to be overridden if not needed
    static Value callConstructor(ContextRef ctx, ValueWeak funcObj, ValueWeak target, std::vector<ValueWeak> args) {
        throw Exception::create(ctx, Exception::Type::TypeError, "Class cannot be called as a constructor");
    }
};

struct MyBuilder3 : public ProtoBuilder::Opaque<MyClass>, public ProtoBuilder::Properties {
    static MyClass* constructOpaque(ContextRef ctx, std::vector<ValueWeak> args) {
        return new MyClass();
    }

    // the default implementation - does not have to be overridden if not needed
    static void destructOpaque(JSRuntime* rt, MyClass* ptr) {
        delete ptr;
    }

    static void addProperties(ContextRef ctx, Object proto) {
        addPropMember<int, &MyClass::foo>(ctx, proto, "foo");
    }
};
```

  2. The `Class` template can be instantiated with the `ProtoBuilder` struct to create a class definition and a name can be assigned using the `init` method. The `init` method can be called repeatedly without any effect if called with the same name; otherwise, an exception will be thrown. The following code shows some examples:

```cpp
using MyClass = Class<MyBuilder>;

MyClass::init("MyClass");
```

  3. To use the class in a given Context, the Context must be initialized with the class definition:

```cpp
ContextRef ctx = ...;

MyClass::initContext(ctx);
```

  4. The user can now get the class constructor and prototype and instantiate the class:

```cpp
Value constructor = MyClass::getConstructor(ctx);
Value prototype = MyClass::getPrototype(ctx);

Value obj = constructor.to<Function>().callConstructor();
```

\subsection{Creating new Features}

An MFeature is defined as a template class. To allow building the inheritance chain of a Machine, the class must set the class `Next` presented in its first template parameter as its base class. To add functionality to the machine, the class may:

  - present a public C++ interface to MFeatures higher in the inheritance chain or to user of the Machine
  - use the `initialize` method to add functionality to the JavaScript runtime

In the MFeature constructor, the user must not interact with the JavaScript runtime in any way, as it is not initialized yet. Initialization of the MFeature involving the JavaScript runtime should be done in the `initialize` method.

The following code shows an example of a Feature that adds a `foo` property to the global object:

```cpp
template<typename Next>
class MyFeature : public Next {
    void initialize() {
        ContextRef ctx = this->_context;
        Object global = ctx.getGlobalObject();
        global.defineProperty("foo", Value::from(ctx, 42));
    }
};
```

The user might also want to define a JavaScript module. This can be done using the `MachineBase::newModule` method:

```cpp
template<typename Next>
class MyFeature : public Next {
    void initialize() {
        ContextRef ctx = this->_context;

        Module& module = this->newModule("myModule");
        module.addExport("foo", Value::from(ctx, 42));
    }
};
```


\shorthandon{'}
\end{markdown*}


\shorthandoff{'}
\begin{markdown*}{%
  hybrid,
  definitionLists,
  footnotes,
  inlineFootnotes,
  hashEnumerators,
  fencedCode,
  citations,
  citationNbsps,
  pipeTables,
  tableCaptions,
}

\chapter{Jaculus-link} \label{chap:link}

As described in the introduction, the Jaculus device and the host communicate through a single byte stream connection.

Because the services running on the device work mostly independently, it is desirable to have independent communication channels for each of them. This can be achieved by multiplexing multiple channels on a stream connection.

Sometimes, the device may have multiple communication interfaces, which can all be used for communication with the host. In such cases, it is desirable to be able to route the communication to the appropriate interface.

This functionality is implemented in the Jaculus-link library, which provides a way to multiplex 256 channels on a single byte stream connection and to route the communication between the appropriate interface and its consumer.

The library is implemented strictly using only the C++ 20 standard library for easy portability. For this reason, it does not provide an implementation for communication interfaces, and the user must provide one themselves.

\section{Architecture}

The model of Jaculus-link is split into three layers:

1. Data link layer
2. Routing layer
3. Communicator layer

\subsection{Data link}

The data link is responsible for transmitting data along with channel identifiers. The data link provided in this library is implemented in the `Mux` class and multiplexes 256 channels on a stream connection.

It is possible to use other data link implementations as long as they implement the `DataLinkTx` interface for transmission and provide a way to connect them to a `DataLinkRx` for processing received data.

\subsection{Routing layer}

The routing layer is responsible for routing the received data to the channel consumer. The routing layer is implemented in the `Router` class.

A `Router` instance can be connected to multiple data links and will route data from all of them to the appropriate consumer with the information about the link it was received from. It also allows sending data to a specific channel and link.

\subsection{Communicator layer}

The communicator layer is used as an abstraction layer for communicating through channels. Typically, the communicator is associated with a single channel and provides either an interface for sending or receiving data.

Communicators used for receiving data from a `Router` must implement the `Consumer` interface, which allows them to be subscribed to a specific channel on a `Router` instance. They must process the received data without blocking, preferably only by storing it in a buffer and processing it later.

Communicators that send data do not have a unified binding interface. Instead, they access the `Router` instance directly and send data to a specific channel on a specific link (or links).

\subsection{Full Jaculus-link pipeline}

An example configuration of the entire pipeline provided by the library is shown in the diagram in Figure \ref{fig:link-pipeline}.

\begin{figure}[!ht]
    \centering
    \includegraphics[width=\textwidth]{img/link-pipeline}
    \caption{An example configuration of the full Jaculus-link pipeline}
    \label{fig:link-pipeline}
\end{figure}


\section{Multiplexer protocol} \label{sec:mux-protocol}

The library provides one protocol for the multiplexer in the class `CobsEncoder`. The protocol is based on a modified version of the COBS\cite{cobs} algorithm for data framing and was originally proposed\footnote{The discussion can be found at \url{https://github.com/yaqwsx/Jaculus/pull/15}.} by Jaroslav Malec for the use in the predecessor version of the Jaculus project.

In the original COBS algorithm, a delimiter byte --- typically zero --- is inserted at the end of the data frame. To encode the transmitted data, every occurrence of the delimiter byte in the data is replaced by a value representing the number of bytes to the next delimiter byte. This allows for data framing with a fixed two-byte overhead while limiting the maximum data size to 254 bytes.

The modified version of the algorithm moves the delimiter byte to the start of the data frame and adds length information to the second byte of the data frame. The rest of the data frame is encoded in the same way as in the original algorithm, except for the missing delimiter byte at the end of the data frame, which is only implied by the data frame length. Every such data frame has a three-byte overhead and can contain up to 254 bytes of data.

Moving the delimiter byte to the start of the data frame allows for resetting the packetization state in case a previous data frame is lost, malformed, or other corrupted data is received on the stream. This is important for the use on microcontrollers, where the used serial port is also often used for logging errors by the built-in libraries. These errors would otherwise cause the packetizer to lose synchronization and cause data frames to be lost.

Figure \ref{fig:cobs-diagram} shows a data frame structure diagram. In the data section of the data frame, the first byte contains the channel identifier, followed by the transmitted data. The last two bytes of the data frame contain the CRC16 checksum of the data section.


\begin{figure}[!ht]
    \centering
    \includegraphics[width=\textwidth]{img/cobs-diagram}
    \raggedright
    \footnotesize{Adapted from: MALEC, Jaroslav. *Protocol diagram*. In: GitHub \[online\]. 2021 \[visited on 2023-05-10\]. Available from: \url{https://github.com/yaqwsx/Jaculus/pull/15}.}
    \caption{Diagram of the COBS-based multiplexer protocol}
    \label{fig:cobs-diagram}
\end{figure}


\section{Usage} \label{sec:link-usage}

\subsection{Adding the library to a project}

Jaculus-link is a header-only library, so adding the `include` directory to the project's include path is sufficient. It is also configured as a CMake project and exports a target `jac-link` that can be linked to other projects.

\subsection{Multiplexer}

The class `Mux` is a data link implemented as a multiplexer and is templated by the class implementing the underlying multiplexer protocol. The only protocol provided with the library is in the `CobsEncoder` class. To implement another protocol, the user can create a class with the same interface as `CobsEncoder`.

The `Mux` constructor takes a `Duplex` instance, which defines the stream connection used for transmitting and receiving data. The `Duplex` interface serves as an abstraction layer for the stream connection and must be implemented by the user.

\subsection{Router}

The `Router` class implements the routing layer. A `Router` instance can be connected to multiple data links, as shown in the following example:

```cpp
// Create a router
Router router;

// Create a stream connection
auto stream = std::make_unique<MyStream>();

// Configure a data link
Mux<CobsEncoder> mux(std::move(stream));

// Connect the data link to the router
auto handle = router.subscribeTx(1, mux);
mux.bindRx(std::make_unique<decltype(handle)>(std::move(handle)));

// Connect another data link to the router
auto link2 = ...;
auto handle2 = router.subscribeTx(2, link2);
link2.bindRx(std::make_unique<decltype(handle2)>(std::move(handle2)));
```

The `handle` object is used to receive data from the data link and must be bound to the same data link instance as the one used to subscribe to the router, as it adds the information about the data link to the received data.

\subsection{Communicators}

The communicators are used as an abstraction layer for communicating through channels. Communicators provide an interface for sending or receiving data through a channel.

The provided communicator types are:

- `OutputStreamCommunicator` --- sends data as a stream of bytes
- `InputStreamCommunicator` --- receives data as a stream of bytes
- `OutputPacketCommunicator` --- sends data while exposing the underlying data framing
- `InputPacketCommunicator` --- receives data while exposing the underlying data framing

These communicator types are only interfaces, and their implementations for `Router` are provided in classes with the same names prefixed with `Router`.

The following example shows how to create a pair of stream communicators:

```cpp
Router router;

// Create an input stream communicator
RouterInputStreamCommunicator input({});

// Subscribe the communicator to the router
router.subscribeChannel(1, input);

// Create an output stream communicator and connect it to the router
RouterOutputStreamCommunicator output(router, 1, {});
```


\shorthandon{'}
\end{markdown*}


\shorthandoff{'}
\begin{markdown*}{%
  hybrid,
  definitionLists,
  footnotes,
  inlineFootnotes,
  hashEnumerators,
  fencedCode,
  citations,
  citationNbsps,
  pipeTables,
  tableCaptions,
}

\chapter{Jaculus-dcore}

To make porting Jaculus to different platforms easier, the core functionality of a Jaculus device (e.g., communication, control protocol, JavaScript runtime) is implemented in the Jaculus-dcore library in a platform-independent way. The library requires C++20 and POSIX support to be available on the target platform.

Jaculus-dcore uses the Jaculus-machine library for the JavaScript runtime and the Jaculus-link library for communication.

\section{Architecture}

The Jaculus-dcore library is centered around the `Device` class, which is used to define a Jaculus device and which bundles all of the core functionality of the library.

\subsection{Device class}

The `Device` class is the entry point for defining a Jaculus device. It is a template parametrized by the Machine type used for the JavaScript runtime. The Machine type must implement the `evalFile` method, which is used to run the code uploaded to the device.

The `Device` class exposes a `Router` object from Jaculus-link, which is used to connect the device to the communication interface. `Device` also exposes an interface for controlling the internal Machine instance from C++ code.

Three services are also part of the `Device` class and expose functionality over the communication channel (or channels):

  - Controller --- service for controlling and monitoring the device
  - Uploader --- service for uploading code/data to the device
  - Logger --- service for logging messages from the device

The services use separate channels provided by the `Router` object. Other channels are also reserved for the standard input and output of the JavaScript instance.

The `Device` implements a locking mechanism to prevent multiple clients from accessing the device simultaneously. The lock is paired with a timeout, which is continuously reset while the client communicates with the devices. If the timeout expires, the lock is released, and other clients can access the device. The lock is exposed to the client via the Controller service.


\section{Implementation}

\subsection{Controller service}

The Controller service is implemented in the `Controller` class. It uses a *PacketCommunicator* interface to communicate with the client.

The first byte of each packet is used to specify the command to be executed. The rest of the packet is used to transmit command-specific data. The packet structure of the protocol is shown in a diagram in Figure \ref{fig:controller-protocol}.

The service exposes the following functionality:

  - accessing the device lock,
  - controlling the internal Machine instance,
  - using the Machine instance's standard input and output, and
  - monitoring the device status.

\begin{figure}[!ht]
    \centering
    \includegraphics[width=0.5\textwidth]{img/controller-packet}
    \caption{Controller protocol packet structure}
    \label{fig:controller-protocol}
\end{figure}

\subsection{Uploader service}

The Uploader service is implemented in the `Uploader` class. Similarly to the Controller service, it uses a *PacketCommunicator* interface to communicate with the client. The first byte of each packet is used to specify the command to be executed. The rest of the packet is used to transmit command-specific data. The packet structure of the protocol is shown in a diagram in Figure \ref{fig:uploader-protocol}.

The service provides the following functionality:

  - listing files and directories,
  - creating and deleting directories, and
  - writing, reading, and deleting files.

Most commands are processed in a single packet, but writing files requires the data to be sent in multiple packets. When writing a file, an internal state is set to specify what operation should be performed when data is received and when the transmission is finished. To prevent overflow of the receiving buffer, the command implements, admittedly relatively inefficient, flow control --- the client must acknowledge each packet before the next one is sent.

Commands for reading a file and listing a directory might also split the data into multiple packets. For simplicity, no flow control is implemented when transmitting data to the client, as the client is expected to be a much more powerful device that can handle the transmission speed.

\begin{figure}[!ht]
    \centering
    \includegraphics[width=0.5\textwidth]{img/controller-packet}
    \caption{Uploader protocol packet structure}
    \label{fig:uploader-protocol}
\end{figure}

\subsection{Filesystem access}

The Uploader service provides access to the filesystem of the device. Unfortunately, because `std::filesystem` is not yet fully implemented in ESP-IDF, the implementation has to, in some cases, rely on the POSIX filesystem API to run on the ESP platform.


\section{Usage}

\subsection{Creating a new device}

To create a new device, an instance of the `Device` class must be created. The class is templated by the Machine type used for the JavaScript runtime. The definition of a Machine type is described in Section \ref{sec:machine-usage}.

The `Device` must be connected to a communication interface. The object exposes a `Router` object from Jaculus-link, which should be used to connect the device to a data link. Definition and binding of a data link are described in Section \ref{sec:link-usage}.

After the device is initialized, the `Device::start` method must be called to start the provided services.

An example of a device definition can be seen in Jaculus-esp32 in the `main.cpp` file.

\subsection{Controlling the device}

To control the device, the user can use the Jaculus-tools application, which provides a command-line interface for managing the device. The application is described in the following chapter in more detail.


\shorthandon{'}
\end{markdown*}


\shorthandoff{'}
\begin{markdown*}{%
  hybrid,
  definitionLists,
  footnotes,
  inlineFootnotes,
  hashEnumerators,
  fencedCode,
  citations,
  citationNbsps,
  pipeTables,
  tableCaptions,
}

\chapter{Jaculus-tools}

To allow the user to interact with Jaculus devices, a command-line application called Jaculus-tools was created. The application is implemented in TypeScript and is available as an npm package under the name `jaculus-tools`.

The package could also be used as a library for building custom applications, but it is not documented. The user might want to look at the source code of the command-line application part of the package for inspiration on how to use the library.

\section{Features}

The application provides commands to check the status of the device (`status`, `version`), install the Jaculus firmware (`install`), control the JavaScript runtime (`start`, `stop`, `monitor`), and access the device's filesystem (`ls`, `read`, `write`, `rm`, `mkdir`, `rmdir`, `upload`). The application also provides commands for uploading and downloading code to the device (`build`, `flash`, `pull`).

Other utility commands are also available:

  - `help` -- prints help for the specified command
  - `list-ports` -- lists available serial ports
  - `serial-socket` -- tunnel a serial port over a TCP socket

The commands are run by specifying them after the `jac` command.

```
$ jac <command> [options] [arguments]
```

\section{Implementation}

\subsection{Connection to the device}

The application implements a simplified version of the interface described in ((TODO - Jaculus-link)) -- the `Router` class is omitted, as connecting to multiple devices is mostly pointless. Aside from that and language choice, the implementation is almost identical.

\subsection{Device access}

The application provides access to the device via the `Device` class. The class exposes the following functionality to the user:

  - device lock
  - Controller and Uploader services
  - standard input and output of the running program
  - output of the device logger

\subsection{Command-line argument parser}

The application uses a custom parser for command-line arguments, which allows for chaining compatible commands.

The commands can also access and modify a global state object passed to each command, allowing them to share data --- for example, once the device is connected, the `Device` instance is saved to the state object and can be accessed by other commands.

For example, the following command:

```
$ jac --port /dev/ttyUSB0 build flash monitor
```

Will sequentially run the three specified commands (`build`, `flash`, `monitor`). The `build` command compiles the code and saves the compiled code to `build` directory. The `flash` command connects to the device, saves the device to the parser state, and uploads the code. The `monitor` command uses the saved device to access its standard input and output without reconnecting.

Command-line options are divided into two types -- global and command-specific. Options can be specified in any place of the command, as the parsing is done in multiple passes for each specified command. The global options are parsed first, then options of the first command are extracted, then of the second, and so on. Each command then receives only its and the global options. This forces the commands to use different names for their options, as otherwise, preceding commands might extract them. For example, the following two commands would be parsed the same way:

```
$ jac --port /dev/ttyUSB0 build flash monitor
$ jac build flash monitor --port /dev/ttyUSB0
```

The commands can also specify standard arguments, which are parsed after the options:

```
$ jac --port /dev/ttyUSB0 read ./code/index.js
```


\subsection{TypeScript code compilation}


\shorthandon{'}
\end{markdown*}


\shorthandoff{'}
\begin{markdown*}{%
  hybrid,
  definitionLists,
  footnotes,
  inlineFootnotes,
  hashEnumerators,
  fencedCode,
  citations,
  citationNbsps,
  pipeTables,
  tableCaptions,
}

\chapter{Jaculus-esp32}

By adding hardware bindings to Jaculus-dcore, it is possible to create firmware for a specific platform. The version created as a part of this thesis is Jaculus-esp32.

Jaculus-esp32 is a Jaculus device firmware for the ESP32 and ESP32-S3 microcontrollers. It uses the ESP-IDF framework and supports connection to a computer via a serial port.

\section{Features}

Aside from the features provided by Jaculus-dcore, Jaculus-esp32 only implements control over the most basic peripherals, which are:

  - GPIO -- general-purpose input/output pins
  - ADC -- analog-to-digital converter
  - LEDC -- generator of PWM signals
  - Neopixel -- WS2812B smart LED strip

Jaculus-esp32 also implements a specialized event queue based on a FreeRTOS queue and supports scheduling events from an interrupt context. This is used in the GPIO feature, which generates events when the state of a pin changes through an interrupt.

Implementation of the Neopixel MFeature uses the SmartLeds\cite{smartleds} library. The library also allows the control of other types of smart LED strips, but for simplicity, only the WS2812B type is supported by the Neopixel MFeature. The MFeature implementation also nicely demonstrates the use of Jaculus-machine for wrapping existing C++ libraries.

\section{Usage}

The firmware can be flashed to the device manually using ESP-IDF or the Jaculus-tools application described in the next chapter.

The firmware uses the Jaculus-machine runtime at its core, meaning that adding new features to the runtime is done the same way described in Chapter \ref{chap:machine}.

\subsection{JavaScript API}

The runtime exposes the following modules to the JavaScript code:

  - `stdio` --- standard input and output
  - `fs` --- file system
  - `path` --- lexical path manipulation
  - `gpio` --- GPIO pins
  - `adc` --- analog-to-digital converter
  - `ledc` --- PWM generator
  - `neopixel` --- WS2812B smart LED strip

\noindent
Aside from these modules, several global constants and functions are also available:

  - `PlatformInfo` --- information about the platform
  - `sleep`, `setTimeout`, `clearTimeout`, `setInterval`, `clearInterval` --- functions for configuring timers
  - `console` --- console object for logging messages
  - `exit` --- function for exiting the program

\noindent
While there is no browsable documentation, the API is documented in more detail in the type definitions files in the TypeScript examples directory attached to the thesis. The examples also demonstrate the usage of some of the modules.

\shorthandon{'}
\end{markdown*}


\shorthandoff{'}
\begin{markdown*}{%
  hybrid,
  definitionLists,
  footnotes,
  inlineFootnotes,
  hashEnumerators,
  fencedCode,
  citations,
  citationNbsps,
  pipeTables,
  tableCaptions,
}


\chapter{Evaluation}

Several requirements were outlined in the introduction. This section evaluates the solution based on those requirements.

Compared to native programs, the solution significantly reduces the length of the development cycle. While the development cycle of native programs can, in worse cases, take several minutes, with Jaculus, it is reduced to several seconds. The build time shows the largest improvement, as no build process is needed for JavaScript programs, and the build time of TypeScript programs is mostly negligible in comparison with native programs. Deployment is also faster, as only the application code is uploaded to the device, in contrast to native programs, where the entire firmware partition is usually overwritten.



  - Minimum C/C++ (e.g., ESP-IDF)
  - Arduino (C/C++)
  - NodeMCU (Lua)
  - CircuitPython (Python)
  - Espruino (Javascript)

  - performance
  - memory usage
  - extensibility
  - usability
  - features


\chapter{Conclusion}

The goal of this thesis was to create an ecosystem for programming embedded devices using JavaScript.

The solution consists of Jaculus-dcore library and Jaculus-tool command-line application. The library provides the core functionality of Jaculus devices, and the application is used to interact with the devices. The Jaculus-dcore library is also integrated into the Jaculus-esp32 firmware, which ports the solution to the ESP32 platform.

Two standalone libraries were also created as part of the solution: Jaculus-link and Jaculus-machine. The former is a communication library, and the latter is an implementation of the JavaScript runtime with easy extensibility. Both libraries are well documented and tested, and can be used independently of the rest of the solution.





\shorthandon{'}
\end{markdown*}


\shorthandoff{'}
\begin{markdown*}{%
  hybrid,
  definitionLists,
  footnotes,
  inlineFootnotes,
  hashEnumerators,
  fencedCode,
  citations,
  citationNbsps,
  pipeTables,
  tableCaptions,
}

\appendix

\chapter{Attachments}

All source code created as a result of this thesis is available in the attached ZIP archive. The archive contains the following directories, each containing a snapshot of a git repository of the respective part of the implementation:

  - `Jaculus-dcore`
  - `Jaculus-machine`
  - `Jaculus-link`
  - `Jaculus-esp32`
  - `Jaculus-tools`
  - `QuickJS`

The `Jaculus-esp32/ts-examples` directory contains examples of TypeScript programs that can be run on the Jaculus-esp32 firmware.

Documentation for `Jaculus-machine` and `Jaculus-link` is available in the `docs` subdirectory of their respective directories. The project's homepage, with Getting Started and Troubleshooting guides, is available in the `docs` subdirectory of the `Jaculus-dcore` directory. The homepage and the documentation are also hosted online at:

  - \url{https://jaculus.org}
  - \url{https://machine.jaculus.org}
  - \url{https://link.jaculus.org}

An upstream version of the respective repositories is available on GitHub:

  - \url{https://github.com/cubicap/Jaculus-machine},
  - \url{https://github.com/cubicap/Jaculus-link},
  - \url{https://github.com/cubicap/Jaculus-tools},
  - \url{https://github.com/cubicap/Jaculus-dcore},
  - \url{https://github.com/cubicap/Jaculus-esp32},
  - \url{https://github.com/cubicap/quickjs}


\chapter{Building and flashing Jaculus-esp32}

The Jaculus-esp32 firmware can be built using the ESP-IDF version 5.0.1 or newer. For the setup of the ESP-IDF, please refer to the official documentation\footnote{In: ESP-IDF Programming Guide \[online\]. \[visited on 2023-05-16\]. Available from: \url{https://docs.espressif.com/projects/esp-idf/en/v5.0.1/esp32/get-started/}}.

The ESP-IDF can be used from the command line. To set up the environment variables for the ESP-IDF, run the corresponding export script in the ESP-IDF directory:

- Linux: `source export.sh`
- Windows: `export.bat` or `export.ps1`

Select the target platform by renaming the corresponding configuration file in the `Jaculus-esp32` directory of the project to `sdkconfig`:

- ESP32: `sdkconfig-esp32`
- ESP32-S3: `sdkconfig-esp32s3`

To build the firmware, run `idf.py build` in the `Jaculus-esp32` directory. To flash the firmware to the device, run `idf.py flash`.

Note that the build process needs to fetch the Jaculus-dcore, Jaculus-machine, Jaculus-link, and QuickJS dependencies from GitHub and requires an internet connection. To install the dependencies manually, copy their source directories to the `Jaculus-esp32/components` directory:

- \texttt{Jaculus-dcore/src} \rightarrow \texttt{components/jaculus-dcore}
- \texttt{Jaculus-machine/src} \rightarrow \texttt{components/jaculus-machine}
- \texttt{Jaculus-link/src} \rightarrow \texttt{components/jaculus-link}
- \texttt{quickjs} \rightarrow \texttt{components/quickjs}


\chapter{Building Jaculus-tools}

Jaculus-tools is developed in TypeScript and requires Node.js version 18 or newer. To build the application, first install its development and runtime dependencies by running `npm ci` in the `Jaculus-tools` directory. Then, build the application by running `npm run build`.

The application can be run using `npx jac` in the `Jaculus-tools` directory. To install the application globally, run `npm link` in the `Jaculus-tools` directory. This will create a symbolic link to the application in the global `node_modules` directory, allowing it to be run from anywhere using `npx jac` (or `jac` if the global `node_modules` directory is in the `PATH` environment variable).


\shorthandon{'}
\end{markdown*}


\end{document}
