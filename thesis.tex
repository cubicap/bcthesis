\documentclass[
  digital,
  oneside,
  nosansbold,
  nocolorbold,
  lof,
  lot
]{fithesis4}

\usepackage[resetfonts]{cmap}
\usepackage[T1,T2A]{fontenc}
\usepackage[
  main=english,
  english, german, czech, slovak
]{babel}

\usepackage{paratype}

\thesissetup{
    date        = \the\year/\the\month/\the\day,
    university  = mu,
    faculty     = fi,
    type        = bc,
    department  = Department of Computer Science,
    author      = Petr Kubica,
    gender      = m,
    advisor     = {RNDr. Jan Mrázek.},
    title       = {Jaculus: Approachable Programming of Embedded Devices via Javascript},
    TeXtitle    = {Jaculus: Approachable Programming of Embedded Devices via Javascript},
    keywords    = {keyword1, keyword2, ...},
    TeXkeywords = {keyword1, keyword2, \ldots},
    abstract    = {
      This is the abstract of my thesis, which can

      span multiple paragraphs.
    },
    thanks      = {
      These are the acknowledgements for my thesis, which can

      span multiple paragraphs.
    },
    bib         = bibliography.bib,
    facultyLogo = fithesis-fi,
}
\usepackage{makeidx}
\makeindex
\usepackage{paralist}
\usepackage{amsmath}
\usepackage{amsthm}
\usepackage{amsfonts}
\usepackage{url}
\usepackage{markdown}
\usepackage{listings}
\lstset{
  basicstyle       = \ttfamily,
  identifierstyle  = \color{black},
  keywordstyle     = \color{blue},
  keywordstyle     = {[2]\color{cyan}},
  keywordstyle     = {[3]\color{olive}},
  stringstyle      = \color{red},
  commentstyle     = \itshape\color{teal},
  breaklines       = true,
  showstringspaces = false,
}
\usepackage{floatrow}
\floatsetup[table]{capposition=top}
\usepackage[babel]{csquotes}


\lstdefinelanguage{js}{
  keywords={typeof, new, true, false, catch, function, return, null, catch, switch, var, if, in, while, do, else, case, break},
  ndkeywords={class, export, boolean, throw, implements, import, this},
  comment=[l]{//},
  morecomment=[s]{/*}{*/},
  morestring=[b]',
  morestring=[b]",
}

\lstdefinelanguage{cpp}{
  keywords={alignas, alignof, and, and_eq, asm, atomic_cancel, atomic_commit, atomic_noexcept, auto, bitand, bitor, bool, break, case, catch, char, char8_t, char16_t, char32_t, class, compl, concept, const, consteval, constexpr, constinit, const_cast, continue, co_await, co_return, co_yield, decltype, default, delete, do, double, dynamic_cast, else, enum, explicit, export, extern, false, float, for, friend, goto, if, inline, int, long, mutable, namespace, new, noexcept, not, not_eq, nullptr, operator, or, or_eq, private, protected, public, reflexpr, register, reinterpret_cast, requires, return, short, signed, sizeof, static, static_assert, static_cast, struct, switch, synchronized, template, this, thread_local, throw, true, try, typedef, typeid, typename, union, unsigned, using, virtual, void, volatile, wchar_t, while, xor, xor_eq},
  comment=[l]{//},
  morecomment=[s]{/*}{*/},
  morecomment=[l][\color{violet}]{\#},
  morestring=[b]',
  morestring=[b]",
}


\begin{document}
\shorthandoff{'}
\begin{markdown*}{%
  hybrid,
  definitionLists,
  footnotes,
  inlineFootnotes,
  hashEnumerators,
  fencedCode,
  citations,
  citationNbsps,
  pipeTables,
  tableCaptions,
}

\chapter{Motivation}

Microcontrollers in embedded devices are typically programmed in compiled languages, such as C, C++, or Rust. Although these languages provide high performance and low overhead at runtime, they can be difficult to learn and use. Another problem of compiled languages in embedded environments is often a long build and deployment time, as the final executable usually contains the full firmware, including all built-in libraries. This, in turn, causes a long development cycle and idle time for the developer.

Most embedded applications have reactive and asynchronous elements, which are difficult to express in low-level languages. This induces a lot of boilerplate code, which obfuscates the main application logic and makes it harder to develop and maintain. Even though high-level languages such as C++ and Rust somewhat alleviate this problem, C is often the only language with direct support from the manufacturers, which again adds more work for the developer.

\chapter{Proposed solution}

The proposed solution is to use a high-level, interpreted language instead.

JavaScript was chosen for its popularity and low hardware requirements of some of its implementations. It also maps very well to the event-centered nature of many embedded systems, as JavaScript is inherently event-driven.

Because JavaScript is weakly typed, debugging errors caused by type mismatches is sometimes challenging. A possible solution is to use a strongly typed language, such as TypeScript. However, as compiling or running TypeScript on a microcontroller is not feasible, it must first be compiled to JavaScript.


\chapter{Approach}

A JavaScript engine is needed to interpret JavaScript code. The chosen engine is QuickJS, a small, embeddable C implementation of the ECMAScript 2020 specification. Compared to other popular options -- DukTape, MujJS, and XS -- it stands on a nice middle ground regarding the feature set, performance, and memory footprint.

The whole solution is split into multiple components:

  - Jaculus-machine -- standalone, embeddable, C++ centric JavaScript runtime using QuickJS at its core
  - Jaculus-link -- standalone communication library for multiplexing multiple channels on a single stream connection
  - Jaculus-device-core -- core library for creating new Jaculus devices
  - Jaculus-tools -- tools for controlling and monitoring Jaculus devices
  - Jaculus-esp32 -- Jaculus device port for the ESP32 platform (with verified support for ESP32 and ESP32-S3)


\chapter{Overview}

\subsection{JavaScript}

JavaScript is a high-level, interpreted, weakly and dynamically typed programming language. It is standardized in the ECMAScript specification, which is maintained by Ecma International.

Although JavaScript programs are event-driven, the code is executed in a single thread. This is achieved by using an event loop, where asynchronous events are queued and executed in the order they are received. Therefore, JavaScript programs must be written in a non-blocking manner, as blocking the event loop will cause the program to stop responding to events. Events are generated by the JavaScript engine or the host environment synchronously and asynchronously.


\subsection{TypeScript}

TypeScript is a strongly typed superset of JavaScript. It is developed and maintained by Microsoft. Typescript is compiled to JavaScript and can be used in any JavaScript environment that supports the selected ECMAScript version.


\subsection{QuickJS}

QuickJS is a JavaScript engine implementing the ECMAScript 2020 specification. It was developed by Fabrice Bellard and is licensed under the MIT license. It is written in C and is designed to be embeddable in other applications.

QuickJS uses POSIX for atomic operations and system time. Although this slightly limits its portability. Fortunately, ESP-IDF, the main target platform for Jaculus, supports POSIX. For supporting other platforms, a custom implementation of the interface would have to be provided.

JavaScript program is executed in a Realm, which defines the execution environment (e.g., global object and set of built-in objects). QuickJS uses a different term for this concept -- Context, which I will be using throughout this thesis, as it is also used in the source code.


\chapter{Jaculus-machine}

Jaculus-machine is a standalone, embeddable, C++ centric JavaScript runtime based on QuickJS. The main goal of Jaculus-machine is to provide a simple and easy-to-use API for adding features to the runtime.

A large part of Jaculus-machine is a set of C++ classes that wrap the QuickJS API. The classes provide RAII semantics and an easy-to-use API for the most common use cases.

Jaculus-machine uses two core concepts, which I will refer to as *Machine* and *Feature* throughout the rest of this thesis.

  - *Machine* -- an instance of the runtime complete with all the selected *Features*
  - *Feature* -- a component that can be used as a part of a *Machine* and which provides functionality to the runtime or to other *Features*

\section{Architecture}

\subsection{Machine and Features}

*Machine* is defined using stack inheritance from `jac::MachineBase` and selected *Feature* classes. The `jac::MachineBase` class provides only the most basic functionality of the runtime, whereas the *Feature* classes provide additional functionality, such as an event loop, filesystem access, etc.

An example definition of a *Machine* type may look like this:

```cpp
using Machine =
    ModuleLoaderFeature<
    FilesystemFeature<
    StdioFeature<
    BasicStreamFeature<
    jac::MachineBase
>>>>;
```

The stack design of *Machine* allows interfacing with different *Features* in C++ directly without any middleware. Lower-level *Features* are located lower in the stack and implement platform specific functionality, while higher-level *Features* are located higher in the stack and can use the abstraction provided by the lower-level *Features*. This allows for easy portability of higher-level *Features* to other platforms.

*Features*, which export functionality to the JavaScript environment, should also provide the same functionality to other *Features* in a way that is as close to the JavaScript API as possible. When a *Feature* defines a JavaScript module, it should wrap the module functionality in a member class instance, so that other *Features* can access the functionality as `this->moduleName.function()`.


\section{Features}

Jaculus-machine adds a minimum set of features to the runtime but provides tools that make creating new *Features* as easy as possible.

\subsection{Value wrapping and conversion}

The library provides a simple API for wrapping JavaScript values and converting them to and from C++ types. The base class for JavaScript value wrapper is `jac::ValueWrapper` and provides a general API for working with JavaScript values. More specific wrapper classes are derived from `jac::ValueWrapper` and provide additional functionality, such as `jac::ObjectWrapper` and `jac::FunctionWrapper`.

Default conversions for some built-in types are provided, such as `int`, `double`, `std::string`, and `std::vector`. The library also provides a mechanism for defining custom conversions for user-defined (and not-yet-supported built-in) types.

Many of these conversions are done automatically in, among others, getters, setters, and function calls. This allows for wrapping existing functions without having to write conversion code manually.

\subsection{Function wrapping}

The library provides an interface for defining JavaScript functions by wrapping almost any existing callable C++ object. This interface is presented in the form of `jac::FunctionFactory` class. Because variadic functions in C++ are processed at build time, they can not be universally wrapped. For this reason, `jac::FunctionFactory` lets the user define a function with an argument of type `std::vector<jac::ValueWrapper>`, which, when called, will contain the arguments passed to the function.

The created wrapper then automatically performs argument and return value type conversions and wraps any exceptions thrown by the wrapped function in a JavaScript exception.

\subsection{Class wrapping}

The library provides an interface for defining JavaScript classes, which can contain opaque C++ objects. The user must first define a `ProtoBuilder` class, which tells the library how to construct the JavaScript object prototype. The class `jac::Class` can then be used to initialize the class in a given context, construct the JavaScript object or obtain its constructor.

\subsection{Built-in *Features*}

The following *Features* are included with the library (their class names are suffixed with `"Feature"`):

  - EventQueue -- an asynchronous event queue that can be used to schedule events to be executed in the event loop; the events can be scheduled from any thread
  - EventLoop -- an event loop that uses an event queue to execute events in the main thread
  - Filesystem -- an abstraction over the filesystem that provides access to files and directories
  - ModuleLoader -- an implementation of module loader for loading modules from the filesystem (using the `import` statement in JavaScript) and evaluating JavaScript files (using the `eval` function in C++)
  - BasicStream -- an implementation of basic readable and writable stream types
  - Stdio -- a feature that adds `stdin`, `stdout`, and `stderr` streams to the *Machine* and `console` interface only to the runtime
  - Timers -- typical JavaScript timers and a sleep function


\section{Implementation}

- ISO C++ 20, QuickJS - posix

\subsection{Default event loop}

Default event loop implementation is split into two separate *Features*. One implements an event queue, and the other implements the event loop. This allows for easier portability, as the event loop itself can be reused on different platforms, while the event queue can be extended to, for example, support scheduling events from interrupts.

\section{Usage}

\subsection{Adding the library to a project}

The library is configured as a CMake project. It exports a CMake target called `jac-machine` that can be linked to other projects.

\subsection{Creating a *Machine*}

 1. The user must define the the *Machine* type using inheritance from `jac::MachineBase` and selected *Features*, as described in ((TODO)).

 2. The *Machine* type can be used to create an instance of the *Machine*:

```cpp
Machine machine;
```

 3. The user can configure the *Machine* and its *Features* using their provided interface. The runtime is still not initialized, and the user must not interact with it.

 4. After configuration the user must initialize the *Machine*:

```cpp
machine.initialize();
```

 5. The *Machine* is now configured, initialized, and ready to use.

\subsection{Running JavaScript code}

The *Machine* can evaluate JavaScript code from a string using the `eval` method. The `eval` method takes the code to evaluate, the filename of the code (used for error reporting), and a set of flags that control the evaluation context. The following code evaluates a simple "Hello, world!" program:

```cpp
// jac::MachineBase::eval(std::string code, std::string filename, jac::EvalFlags flags)
machine.eval("console.log('Hello, world!');", "<eval>", jac::EvalFlags::Global);
```

The return value of the `eval` method is a `jac::ValueWrapper`, which contains the result of the evaluation. In this case, the result is `undefined` and is ignored.

To evaluate a file, `ModuleLoaderFeature::evalFile` method can be used:

```cpp
// jac::ModuleLoaderFeature::evalFile(std::string filename)
machine.moduleLoaderFeature.evalFile("hello.js");
```

Any unhandled exceptions thrown by the evaluated code will be wrapped to C++ exceptions of type `jac::Exception` and thrown by the `eval` method.

\subsection{Running the event loop}

After evaluating some code, the execution will stop. To continue execution, the user must run the event loop, which can be done using the `EventLoopFeature::runEventLoop` method:

```cpp
machine.runEventLoop();
```

The `runEventLoop` method will block until the event loop is stopped. It must be stopped manually either by calling the `EventLoopFeature::stop`, `EventLoopFeature::kill`, or JavaScript function `exit`.

If an unhandled exception is thrown by the event loop, it will be wrapped to C++ exception of type `jac::Exception` and thrown by the `runEventLoop` method. The event loop will also be stopped.


  - JS value wrapping, conversion
  - class wrapping
  - function wrapping
  - exceptions
  - watchdog

\chapter{Jaculus-link}

\section{Features}

  - multiplexing multiple channels on a single stream connection
  - routing multiple channelized connections to a single consumer

\section{Architecture}

  - *Mux* - multiplexing multiple channels on a single stream connection
  - *Router* - routing multiple channelized connections to a single consumer
  - *Consumer* - interface for receiving data from a channel

\section{Implementation}

  - ISO C++ 20
  - no connection bindings
  - protocol

\section{Usage}

  - defining a Stream
  - configuring Mux
  - configuring Router
  - creating Consumer


\chapter{Jaculus-device-core}

\section{Features}

  - Jaculus-machine runtime
  - communication via Jaculus-link
  - control protocol for controlling and monitoring Jaculus-machine runtime
  - uploader protocol for uploading code/data to Jaculus device

\section{Architecture}

  - *Controller* - entry point for defining a Jaculus device and service for controlling it
  - *Uploader* - service for uploading code/data to Jaculus device
  - *Machine* - Jaculus-machine runtime
  - central locking mechanism with a timeout to prevent multiple clients from accessing the device at the same time

\section{Implementation}

  - ISO C++ 20, posix (esp + std::filesystem, QuickJS)
  - Uploader, Controller protocols

\section{Usage}

  - defining a Jaculus device
  - controlling a Jaculus device


\chapter{Jaculus-tools}

\section{Features}

  - access to Jaculus devices via Uploader and Controller protocols
  - command-line tool
  - Node.js library
  - transpile TypeScript code

\section{Implementation}

  - TypeScript
  - access to Jaculus device services via JacDevice class

\section{Usage}

  - command-line tool
  - Node.js library


\chapter{Jaculus-esp32}

\section{Features}

  - support for basic peripherals (GPIO, ADC, LEDC, Neopixel)

\section{Usage}

  - flash manually using ESP-IDF
  - flash using Jaculus-tools


\chapter{Pitfalls}

\section{Only one context per Machine}
From the beginning, the design of Jaculus-machine was to have only one context per machine. It seemed not to be a limiting factor and simplified implementation. Now it starts to show it was a wrong decision and proves to be a limitation in some cases.

For example, in REPL, all exceptions should be caught and reported to standard output. When starting REPL from a JavaScript program, the main program should crash on unhandled exceptions, whereas the REPL should not. Implementation of this would require REPL and the main program to be executed in separate contexts to distinguish between their behavior regarding exception handling.


\section{Unhandled promise rejections not being reported}
This is a limitation of QuickJS. Although QuickJS does have a mechanism for reporting unhandled promise rejections, it reports some false positives. Consider the following example:

``` js
new Promise((resolve, reject) => {
    console.log("promise");
    reject(null);
}).then(() => {
    console.log("ok");
}).catch(() => {
    console.log("error");
});

console.log("after");
```

The promise is created and immediately rejected. At that moment, the promise does not have a rejection handler, and thus QuickJS reports an unhandled promise rejection. However, the handler is added before the promise goes out of scope and handles the rejection.

Fortunately, this is not a problem for well-written code, which handles all promise rejections. However, when an unhandled promise rejection occurs, it is not reported, which may lead to errors that are difficult to debug.

To fix this, modifying QuickJS internals would be required, which is outside the scope of this work, but it is an essential consideration for future work.


  % - no debugger - QuickJS does not support debugging



\chapter{Comparison with existing solutions}

  - Minimum C/C++ (e.g. ESP-IDF)
  - Arduino (C/C++)
  - NodeMCU (Lua)
  - CircuitPython (Python)
  - Espruino (Javascript)

  - performance
  - memory usage
  - extensibility
  - usability
  - features


\chapter{Future work}

  - Web IDE
  - REPL
  - debugger?
  - add more features
  - support more platforms (e.g. STM32, RPi)

\shorthandon{'}
\end{markdown*}
\end{document}
